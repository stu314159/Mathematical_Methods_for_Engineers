\chapter{Lecture 14 - Modified Bessel Function and Parametric Modified Bessel Function}
\label{ch:lec14}
\section{Objectives}
\begin{itemize}
\item Show how to use the parametric Bessel equation of order $\nu$.
\item Describe the modified Bessel equation and, their solutions, modified Bessel functions.
\item Introduce and illustrate a tool for solving second-order ODEs in terms of Bessel functions.
\end{itemize}

Many differential equations can be solved in terms of Bessel functions.  This lecture will introduce some relatively simple and powerful tools for doing so.

\section{Parametric Bessel Equation of Order $\nu$}
The parametric Bessel equation of order $\nu$ has the form given in Equation \ref{eq:pbe-form}

\begin{equation}
x^{2}u^{\prime \prime} + xu^{\prime} + \left(\alpha^2 x^2 - \nu^2 \right)u= 0
\label{eq:pbe-form}
\end{equation}

Rather than state the solution outright, let us take a different shortcut and apply the well-known transformation that will convert Equation \ref{eq:pbe-form} into Bessel's equation that we can solve, by inspection, with Bessel functions.

\newthought{The transformation is} $t = \alpha x$.  This, of course, means that $x = \sfrac{t}{\alpha}$ and the derivatives of $u$ with respect to $t$ are shown in the margin.\marginnote{The derivatives of $u$ with respect to $t$:
\begin{align*}
\frac{dt}{dx} &= \alpha \\
u^{\prime} &= \frac{du}{dx} = \frac{du}{dt}\frac{dt}{dx} = \alpha u_t \\
u^{\prime \prime} &= \frac{d}{dx}\left(\frac{du}{dx}\right) = \cdots \\
&= \frac{d}{dt}\left(\frac{du}{dx}\right)\frac{dt}{dx} = \cdots \\
&=\frac{d}{dt}\left(\alpha u_t\right)\alpha = \alpha^{2}u_{tt}
\end{align*}
} Applying these substitutions gives us:
\begin{align*}
x^{2}u^{\prime \prime} + xu^{\prime} + \left(\alpha^2 x^2 - \nu^2 \right)u &= 0 \\
\frac{t^2}{\alpha^2}\alpha^2 u_{tt} + \frac{t}{\alpha}\alpha u_t + \left(\alpha^2 \frac{t^2}{\alpha^2} - \nu^2 \right)u &= 0 \\
t^2u_{tt} + t u_{t} + \left(t^2 - \nu^2\right)u &= 0
\end{align*}
The last line is, of course, Bessel's equation and the solution is:\marginnote{At this point in the course you should be developing a list, of sorts, of differential equations that you recognize and know how to analyze.  Call it something like \emph{``A Field Guide to Differential Equations I Know How To Solve''.}  This list should include:
\begin{itemize}
\item first-order linear equations
\item separable equations
\item linear constant-coefficient equations
\item linear equations with variable coefficients, including:
\begin{itemize}
\item Cauchy-Euler equations
\item Legendre's equation
\item Bessel's equation; and now
\item parametric Bessel's equation.
\end{itemize}

\end{itemize} 
For these last problem types you ``solve'' them by recognizing the equation and writing down the solution.} 
\begin{equation*}
u(t) = c_1J_{\nu}(t) + c_2Y_{\nu}(t)
\end{equation*}
Undoing the change of independent variables to express the answer in terms of $x$ gives us the solution shown in Equation \ref{eq:pbe-sol}.
\begin{equation}
u(x) = c_1J_{\nu}(\alpha x) + c_2Y_{\nu}(\alpha x)
\label{eq:pbe-sol}
\end{equation}

\vspace{0.5cm}

\noindent\textbf{Example:} us the parametric Bessel equation to find the general solution to:
\begin{equation*}
x^2u^{\prime \prime} + xu^{\prime} + \left(36x^2 - \frac{1}{4} \right)u = 0
\end{equation*}
We recognize the equation as a parametric Bessel equation; the parameter $\alpha = \sqrt{36} = 6$ and $\nu^2 = \sfrac{1}{4} \Rightarrow \nu = \sfrac{1}{2}$.  The solution is:\marginnote{\textbf{Note: }for the example, instead of using $Y_{\sfrac{1}{2}}(6x)$ as the second linearly independent solution, we could have used $J_{-\sfrac{1}{2}}(6x)$; it is entirely up to you.}
\begin{equation*}
u(x) = c_1 J_{\sfrac{1}{2}}(6x) + c_2Y_{\sfrac{1}{2}}(6x)
\end{equation*}

\section{Modified Bessel Equations and Bessel Functions}
A subtle but non-trivial variation to Bessel's equation is when we flip one crucial sign as shown in Equation \ref{eq:mbeq-form}
\begin{equation}
x^2u^{\prime \prime} + xu^{\prime}-\left(x^2 + \nu^2\right)u = 0
\label{eq:mbeq-form}
\end{equation}
\index{Bessel's Equation, parametric}
\index{Bessel's Equation, modified}
\index{Bessel Function, modified}
This equation can be converted into Bessel's equation by transforming the dependent variable---$u=i^{-\nu}v$---and independent variable---$t=ix$.  We will omit these details and instead simply give the solution as shown in Equation \ref{eq:mbeq-solution} which includes modified Bessel functions of the first\sidenote{Modified Bessel functions of the first kind are defined as:
\begin{equation*}
I_{\nu}(x) = i^{-\nu}J_{\nu}(ix)
\end{equation*}} and second kind\sidenote{Modified Bessel functions of the second kind, analogous to Bessel functions of the second kind, are defined in terms of modified Bessel functions of the first kind:
\begin{equation*}
K_{\nu}(x) = \frac{\pi}{2}\frac{I_{\nu}(x)-I_{\nu}(x)}{\sin{\nu \pi}}
\end{equation*}
}:

\begin{equation}
u(x) = c_1I_{\nu}(x) + c_2K_{\nu}(x)
\label{eq:mbeq-solution}
\end{equation}

The modified Bessel's equation also has a parametric form as shown in Equation \ref{eq:mpbeq-form}
\begin{equation}
x^2u^{\prime \prime} + xu^{\prime} - \left(\alpha^2 x^2 + \nu^2 \right)u = 0
\label{eq:mpbeq-form}
\end{equation}
with the general solution given in Equation \ref{eq:mpbeq-solution}.
\begin{equation}
u(x) = c_1I_{\nu}(\alpha x) + c_2 K_{\nu}(\alpha x)
\label{eq:mpbeq-solution}
\end{equation}

\section{Tool for Solving Second-Order ODEs}
A more general-purpose tool for solving linear, homogeneous, second-order ODEs in terms of Bessel functions is presented in \cite{zill2020advanced} and shown below in Equation \ref{eq:bessel-tool}.

\begin{equation}
u^{\prime \prime} + \frac{1-2a}{x}u^{\prime} + \left(b^2 c^2 x^{2c-2} + \frac{a^2-p^2c^2}{x^2} \right)u = 0, \ \ p \ge 0
\label{eq:bessel-tool}
\end{equation}
The general solution for equations of this form is given in Equation \ref{eq:bessel-tool-sol}.
\begin{equation}
u = x^a\left[c_1 J_{p}\left(bx^c \right)+c_2Y_{p}\left(bx^c \right) \right]
\label{eq:bessel-tool-sol}
\end{equation}
Using this tool requires you to solve four non-linear equations as shown below:\marginnote{Probably the most challenging or, at least, error-prone part of this process is writing a given ODE in the form of Equation \ref{eq:bessel-tool}.}
\begin{equation*}
%u^{\prime \prime} + \frac{\overbracket{\fbox{$1-2a$}}^{\textcircled{1}}}{x}u^{\prime} + \left(\underbracket{\fbox{$b^2 c^2$}}_{\textcircled{2}} x^{\overbracket{\fbox{$2c-2$}}^{\textcircled{3}}} + \frac{\overbracket{\fbox{$a^2-p^2c^2$}}^{\textcircled{4}}}{x^2} \right)u = 0, \ \ p \ge 0
u^{\prime \prime} + \frac{\overset{\textcircled{1}}{\fbox{$1-2a$}}}{x}u^{\prime} + \left(\underset{\textcircled{2}}{\fbox{$b^2 c^2$}} x^{\overset{\textcircled{3}}{\fbox{$2c-2$}}} + \frac{\overset{\textcircled{4}}{\fbox{$a^2-p^2c^2$}}}{x^2} \right)u = 0, \ \ p \ge 0
\end{equation*}

\vspace{0.5cm}

\noindent\textbf{Example:}  Use Equation \ref{eq:bessel-tool} to find the general solution to the following differential equation:
\begin{equation*}
x^2 u^{\prime \prime} + \left(x^2 - 2\right)u = 0
\end{equation*}

Re-writing the equation in the form of Equation \ref{eq:bessel-tool} gives us:
\begin{align*}
x^2 u^{\prime \prime} + \left(x^2 - 2\right)u &= 0 \\
u^{\prime \prime} + \frac{x^2-2}{x^2}u &= 0 \\
u^{\prime \prime} + 0u^{\prime} + \left(1x^0 + \frac{-2}{x^2} \right)u &= 0
\end{align*}

Now we solve the four equations:
\begin{itemize}
\item $\textcircled{1}$  $1-2a = 0 \Rightarrow a = \sfrac{1}{2}$
\item $\textcircled{3}$ $2c-2=0 \Rightarrow c = 1$
\item $\textcircled{2}$ $b^2c^2=1, \ \ b^2(1) = 1, \ \Rightarrow b=1$
\item $\textcircled{4}$ $a^2 - p^2c^2=-2$ 
\begin{align*}
\left(\frac{1}{2} \right)^2 - p^2(1)^2 &=-2 \\
p^2 &= \frac{1}{4} + 2 = \frac{9}{4} \\
\Rightarrow p &= \frac{3}{2}
\end{align*}
\end{itemize}
Using Equation \ref{eq:bessel-tool-sol} the general solution is:
\begin{equation*}
u(x) = x^{\sfrac{1}{2}}\left[c_1 J_{\sfrac{3}{2}}(x) + c_2 Y_{\sfrac{3}{2}}(x) \right]
\end{equation*}


