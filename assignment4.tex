\chapter{Assignment \#4}
\label{ch:ass4}
\begin{fullwidth}
Find two power series solutions of the given differential equation.
\begin{enumerate}
\item $u^{\prime \prime} + x^2 u^{\prime} + xu = 0$

\vspace{1.0cm}

\item $u^{\prime \prime} - (x+1)u^{\prime} - u = 0$
\end{enumerate}

\vspace{1.0cm}



\begin{enumerate}[resume]
\item Solve the given initial value problem.  Use MATLAB to represent the power series.  Make 2 plots; for the first plot compare the partial sum of the power series with 5 terms to the exact solution which is $u=8x-2e^x$; for the second plot compare the partial sum of the power series with 15 terms to the exact solution.  Submit the ``published'' version of your MATLAB script (PDF format) along with your written solution.

\begin{equation*}
(x-1)u^{\prime \prime} - xu^{\prime} + u = 0, \ \ u(0)=-2, \ u^{\prime}(0) = 6
\end{equation*}
\end{enumerate}

\vspace{1.0cm}

\noindent Determine the singular points for the differential equation.  Classify each singular point as irregular or regular.
\begin{enumerate}[resume]
\item $\left(x^2-9 \right)^2u^{\prime \prime} + (x+3)u^{\prime} + 2u = 0$

\vspace{1.0cm}

\item $x^3\left(x^2 - 25\right)\left(x-2\right)^2u{\prime \prime}+3x(x-2)u^{\prime}+7(x+5)u = 0$
\end{enumerate}

\vspace{1.0cm}

\noindent Use the general form of the indicial equation to find the indicial roots.

\begin{enumerate}[resume]
\item $x^2u^{\prime \prime} + \left(\frac{5}{3}x+x^2 \right)u^{\prime}-\frac{1}{3}u = 0$
\end{enumerate}

\vspace{1.0cm}

\noindent Use the method of Frobenius to obtain two linearly independent series solutions:
\begin{enumerate}[resume]
\item $3xu^{\prime \prime} + (2-x)u^{\prime} - u = 0$
\end{enumerate}

\end{fullwidth}
