\chapter{Lecture 6 - Cauchy-Euler Equations}
\label{ch:lec6}
\section{Objectives}
The objectives of this lecture are:
\begin{itemize}
\item Introduce Cauchy-Euler equations and demonstrate a method of solution.
\item Carry out some examples to illustrate the methods for 2\textsuperscript{nd}-order, homogeneous Cauchy-Euler equations.
\end{itemize}

\section{Cauchy-Euler Equations}

A linear differential equation of the form
\begin{equation}
a_nx^n\frac{d^n u}{dx^n}+a_{n-1}x^{n-1}\frac{d^{n-1}u}{dx^{n-1}}+\cdots+a_1x\frac{du}{dx}+a_0u=g(x)
\label{eq:cauchy-euler-eqn}
\end{equation}
is called a Cauchy-Euler equation.

\newthought{Take note of} the relationship between the exponent of $x$ in the coefficients and the order of the differential operators.  This correspondence between the decreasing power of $x$ in the coefficient and the decreasing order of the differential operator is characteristic of this type of equation and is the way you should recognize it.\marginnote[-2.0cm]{It is the corresponding \emph{change} in power/order that matters.  The equation: $a\frac{d^2u}{dx^2}+\frac{1}{x^2}u=0$ is also a Cauchy-Euler equation since the power of $x$ in the coefficient goes from 0 to -2 while the order of the differential operator goes from 2\textsuperscript{nd} to 0.} 

Also observe that this equation is \emph{linear}; if $g(x)=0$ it is homogeneous, otherwise it is non-homogeneous.  For this lecture we will focus our attention on the homogeneous, 2\textsuperscript{nd}-order Cauchy-Euler equation:
\begin{equation}
ax^2\frac{d^2u}{dx^2}+bx\frac{du}{dx}+cu=0
\end{equation}

Lastly you should be alert to the fact that the coefficient for the highest order derivative is 0 at $x=0$; consequently we will restrict the interval of interest for these equations to $x \in \left(0,\infty\right)$.

\vspace{4.0cm}

\newthought{The basic strategy} in solving these equations is to try a solution in the form $u(x)=x^m$.  When we substitute this solution into the equation we get:\marginnote{If $u(x) = x^m$ then, of course, $u^{\prime} = mx^{(m-1)}$ and $u^{\prime \prime} = m(m-1)x^{m-2}$.}
\begin{align*}
am(m-1)x^2x^{m-2}+bmxx^{m-1}+cx^m &=0 \\
x^m\left[am(m-1)+bm+c \right]&=0
\end{align*}
That last part in the brackets is referred to as the ``auxiliary equation'':
\begin{equation}
am^2+(b-a)m+c=0
\label{eq:CE-aux}
\end{equation}
We will look for values of $m$ that satisfy this quadratic equation; those values will be the exponents for our solutions.

\newthought{As is the case} for quadratic equation, there are three possible outcomes: 

\begin{enumerate}
\item \textbf{Distinct Real Roots.} In this case $m_1 \ne m_2$ and the general solution is of the form:
\begin{equation}
u(x)=c_1x^{m_1}+c_2x^{m_2}
\end{equation}

\vspace{0.5cm}

\noindent\textbf{Example:} Find the general solution for $x^2\frac{d^2u}{dx^2}-2x\frac{du}{dx}-4u=0$.

\vspace{0.25cm}

\noindent Referring to Equation \ref{eq:CE-aux}, $a=1,\ b=-2, \ c=-4$\marginnote{Be careful with these coefficients. In contrast to the case with constant coefficient linear equations, we do not plug these coefficients directly into the quadratic equation. Instead we put them in the auxiliary equation and then solve \emph{that} with the quadratic equation.} so the auxiliary equation is:
\begin{align*}
m^2-3m-4 &= 0 \\
(m-4)(m+1) &=0
\end{align*}
By inspection the roots are $m_1=4$ and $m_2=-1$.  The general solution is $u(x)=c_1x^4+c_2x^{-1}$.

\vspace{0.5cm}

\item \textbf{Real Repeated Roots.} In this case, $m_1 = m_2$.  We have one solution, $u_1(x)=c_1x^{m_1}$.  Clearly we need to take some kind of action if we hope to get another linearly independent solution.  It can be shown that if we form the second solution by multiplying the first solution by $\ln{x}$---$u_2(x)=\ln{(x)}u_1(x)$---then $u_2(x)$ will satisfy the governing equation and also be linearly independent from $u_1(x)$.\marginnote[-1.5cm]{The first one or two times you solve these problems, you should verify both of those assertions. Namely that: 
\begin{enumerate}
\item $u_2(x) = \ln{(x)}u_1(x)$ is a solution to the equation; and
\item $u_2(x)$ is linearly independent from $u_1(x)$.
\end{enumerate}}

\vspace{0.5cm}

\noindent\textbf{Example:} Find the general solution for $4x^2 \frac{d^2u}{dx^2}+8x\frac{du}{dx}+u=0$.

\vspace{0.25cm}

\noindent The auxiliary equation in this case is: $4m^2+4m+1=0$.  This can be factored to give $(2m+1)^2=0$ so we have a case of repeated roots where $m_1=m_2=-\frac{1}{2}$.

The solution is: $u(x)=c_1x^{-1/2}+c_2x^{-1/2}\ln{x}$.

\vspace{0.5cm}

\item \textbf{Complex Conjugate Roots.}  This case is completely analogous with the previous cases vis-\`a-vis linear constant coefficient equations.  The roots are $m_{1,2} = \alpha \pm i\beta$ and the general solution is:
\begin{equation}
u(x) = x^{\alpha}\left[c_1 \cos{(\beta \ln{x})} + c_2 \sin{(\beta \ln{x})} \right]
\end{equation}

\vspace{0.5cm}

\noindent\textbf{Example:} Solve: $4x^2 u^{\prime \prime} +17u=0, \ \ u(1)=-1, \ u^{\prime}(1)=-1/2$.

\vspace{0.25cm}

\noindent The auxiliary equation is $4m^2-4m+17=0$.  Using the quadratic formula the roots are found to be: 
\begin{align*}
m_{1,2} &= \frac{4 \pm \sqrt{16-4(4)(17)}}{8} \\
&=\frac{1}{2} \pm \frac{\sqrt{-256}}{8} \\
&=\frac{1}{2} \pm \frac{16i}{8} \\
&=\underbracket{\frac{1}{2}}_{\alpha} \pm \underbracket{2}_{\beta}i
\end{align*}

\noindent So the general solution is:
\begin{equation*}
u(x)=x^{1/2}\left[c_1 \cos{(2 \ln{x})}+c_2 \sin{(2 \ln{x})} \right]
\end{equation*}
We can apply the first boundary condition, $u(1)=-1$:
\begin{align*}
u(1) &= 1 \left[c_1 \cos{0} + c_2 \sin{0} \right] \\
&= c_1(1) + c_2(0) = -1 \\
\Rightarrow c_1&=-1
\end{align*}
The calculus is a bit more tedious for the second boundary condition:

\begin{multline*}
u^{\prime}(x) = -\frac{1}{2}x^{-1/2}\cos{(2 \ln{x})} + 2x^{-1/2}\sin{(2 \ln{x})} + ...\\
c_2\left[ \frac{1}{2} x^{-1/2} \sin{(2 \ln{x})} + 2x^{-1/2}\cos{(2 \ln{x})} \right]
\end{multline*}
Evaluating this at $x=1$:
\begin{align*}
u^{\prime}(1)&=-\frac{1}{2}(1)(1)+2(1)(0)+c_2[0+2(1)(1)] \\
&=-\frac{1}{2}+2c_2 = -\frac{1}{2} \\
\Rightarrow c_2&=0
\end{align*}
So the solution is: $u(x) = -x^{1/2}\cos{(2 \ln{x})}$.

\end{enumerate}

\section{Non-homogeneous Cauchy-Euler Equations}
Sadly, the method of undetermined coefficients will not work with Cauchy-Euler equations, a limitation of that method being that the coefficients need to be constant. Interested students can investigate the method called \emph{variation of parameters} that can be used to address this problem analytically.  Otherwise, we will plan to use numerical methods to solve non-homogeneous problems of this type.

\section{Derivation of the Solution to Cauchy-Euler Equations}
It would be hard not to notice the similarity in the solution methods of Cauchy-Euler equations and constant coefficient linear equations.  This is not a coincidence.  In this section I want to briefly show you that, through a change of variables, Cauchy-Euler equations are, in some sense, equivalent to constant coefficient linear equations.

\subsection{Change of Independent Variable}
What we will do, is change the independent variable from $x$ to $e^t$.\sidenote{Think of this as ``stretching'' the $x$-axis.}  If $x = e^t$, that means that $t = \ln{x}$ and $\frac{dt}{dx} = \frac{1}{x} = e^{-t}$.

If we consider, again, the 2\textsuperscript{nd}-order Cauchy-Euler equation,
\begin{equation*}
ax^2\frac{d^2u}{dx^2}+bx\frac{du}{dx}+cu = 0
\end{equation*}
every appearance of $x$ needs to be converted into its equivalent in terms of $t$ and every derivative with respect to $x$ needs to be converted into a derivative with respect to $t$.

It's easy enough to replace $x$ with $e^t$; converting the derivatives takes a bit more work.  We will use the chain rule as shown below:
\begin{align*}
\frac{du}{dx} &= \frac{du}{dt} \frac{dt}{dx} \\
&=u_t e^{-t}
\end{align*}
where we use the subscript notation to denote derivatives with respect to $t$ and use the substitution $\frac{dt}{dx}=e^{-t}$ as determined above.

We do it again, to convert the second derivatives:
\begin{align*}
\frac{d^2u}{dx^2} &= \frac{d}{dx}\left(\frac{du}{dx}\right) \\
&= \frac{d}{dt}\left(\frac{du}{dx}\right)\frac{dt}{dx} \\
&=\frac{d}{dt}\left(u_t e^{-t}\right)e^{-t} \\
&= \left(u_{tt}e^{-t}-u_te^{-t}\right)e^{-t} \\
&= e^{-2t}\left(u_{tt} - u_{t} \right)
\end{align*}

We are now ready to make our substitutions into the differential equation:

\begin{equation*}
a\underbracket{e^{2t}}_{x^2}\overbracket{e^{-2t}\left(u_{tt}-u_t\right)}^{\frac{d^2u}{dx^2}}+b\underbracket{e^{t}}_{x}\overbracket{u_te^{-t}}^{\frac{du}{dx}}+cu=0
\end{equation*}
Combining terms to simplify gives us Equation \ref{eq:ce-conv} which is now, under this change of variables, a 2\textsuperscript{nd}-order linear constant coefficient equation.

\begin{equation}
au_{tt} + (b-a)u_t + cy = 0
\label{eq:ce-conv}
\end{equation}
If I solve this using our standard method, the resulting auxiliary equation is the same as what is shown in Equation \ref{eq:CE-aux}.

\newthought{In the case} of constant coefficient linear equations, the solutions were of the form $u = e^{mx}$ which, according to the exponentiation rules, the same as $u = e^{x^m}$.  But now, our independent variable is $t$, where $t=\ln{x}$. With this substitution:
\begin{align*}
u(t) &= e^{(\ln{x})^m} \\
&= x^m
\end{align*}
which is the assumed form of solution for Cauchy-Euler equations.


