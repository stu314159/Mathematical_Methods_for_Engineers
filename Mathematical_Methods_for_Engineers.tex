\documentclass{tufte-book}

\hypersetup{colorlinks}% uncomment this line if you prefer colored hyperlinks (e.g., for onscreen viewing)

%%
% Book metadata
\title{Mathematical Methods for Engineers}
\author[CAPT Stu Blair]{United States Naval Academy}
\publisher{Mighty Goat Press}

%%
% If they're installed, use Bergamo and Chantilly from www.fontsite.com.
% They're clones of Bembo and Gill Sans, respectively.
%\IfFileExists{bergamo.sty}{\usepackage[osf]{bergamo}}{}% Bembo
%\IfFileExists{chantill.sty}{\usepackage{chantill}}{}% Gill Sans

%\usepackage{microtype}

%%
% Just some sample text
\usepackage{lipsum}

%%
% For nicely typeset tabular material
\usepackage{tabularx}
\usepackage{booktabs}

%%
% For graphics / images
\usepackage{graphicx}
\setkeys{Gin}{width=\linewidth,totalheight=\textheight,keepaspectratio}
\graphicspath{{graphics/}}

% The fancyvrb package lets us customize the formatting of verbatim
% environments.  We use a slightly smaller font.
\usepackage{fancyvrb}
\fvset{fontsize=\normalsize}

%%
% Prints argument within hanging parentheses (i.e., parentheses that take
% up no horizontal space).  Useful in tabular environments.
\newcommand{\hangp}[1]{\makebox[0pt][r]{(}#1\makebox[0pt][l]{)}}

%%
% Prints an asterisk that takes up no horizontal space.
% Useful in tabular environments.
\newcommand{\hangstar}{\makebox[0pt][l]{*}}

%%
% Prints a trailing space in a smart way.
\usepackage{xspace}

%%%% packages added by Stu

\usepackage{comment}

% fancy enumeration tricks
\usepackage{enumitem}

% have appendices
\usepackage{appendix}

% has sfrac command
\usepackage{xfrac}

% mathtools for over/under brace/bracket among other nifty tools
\usepackage{mathtools}


\usepackage{cancel} % to get oblique strike-through

%\usepackage{minted}
\usepackage{pifont}
\usepackage{color}

\usepackage{listings}
\definecolor{mygreen}{rgb}{0,0.6,0}
\definecolor{mygray}{rgb}{0.5,0.5,0.5}
\definecolor{mymauve}{rgb}{0.58,0,0.82}
\lstset{ %
  backgroundcolor=\color{white},   % choose the background color; you must add \usepackage{color} or \usepackage{xcolor}
  basicstyle=\footnotesize,        % the size of the fonts that are used for the code
  breakatwhitespace=false,         % sets if automatic breaks should only happen at whitespace
  breaklines=true,                 % sets automatic line breaking
  captionpos=b,                    % sets the caption-position to bottom
  commentstyle=\color{mygreen},    % comment style
  deletekeywords={...},            % if you want to delete keywords from the given language
  escapeinside={\%*}{*)},          % if you want to add LaTeX within your code
  extendedchars=true,              % lets you use non-ASCII characters; for 8-bits encodings only, does not work with UTF-8
  frame=single,                    % adds a frame around the code
  inputpath=./matlab_examples/,
  keepspaces=true,                 % keeps spaces in text, useful for keeping indentation of code (possibly needs columns=flexible)
  keywordstyle=\color{blue},       % keyword style
  language=Matlab,                 % the language of the code
  morekeywords={*,...},            % if you want to add more keywords to the set
  numbers=left,                    % where to put the line-numbers; possible values are (none, left, right)
  numbersep=5pt,                   % how far the line-numbers are from the code
  numberstyle=\tiny\color{mygray}, % the style that is used for the line-numbers
  rulecolor=\color{black},         % if not set, the frame-color may be changed on line-breaks within not-black text (e.g. comments (green here))
  showspaces=false,                % show spaces everywhere adding particular underscores; it overrides 'showstringspaces'
  showstringspaces=false,          % underline spaces within strings only
  showtabs=false,                  % show tabs within strings adding particular underscores
  stepnumber=2,                    % the step between two line-numbers. If it's 1, each line will be numbered
  stringstyle=\color{mymauve},     % string literal style
  tabsize=2,                       % sets default tabsize to 2 spaces
  title=\lstname                   % show the filename of files included with \lstinputlisting; also try caption instead of title
}

% example usage for code: \begin{lstlisting}[caption=< caption text >, label=<label>]

%%% end packages added by Stu
%%
% Some shortcuts for Tufte's book titles.  The lowercase commands will
% produce the initials of the book title in italics.  The all-caps commands
% will print out the full title of the book in italics.
\newcommand{\vdqi}{\textit{VDQI}\xspace}
\newcommand{\ei}{\textit{EI}\xspace}
\newcommand{\ve}{\textit{VE}\xspace}
\newcommand{\be}{\textit{BE}\xspace}
\newcommand{\VDQI}{\textit{The Visual Display of Quantitative Information}\xspace}
\newcommand{\EI}{\textit{Envisioning Information}\xspace}
\newcommand{\VE}{\textit{Visual Explanations}\xspace}
\newcommand{\BE}{\textit{Beautiful Evidence}\xspace}

\newcommand{\TL}{Tufte-\LaTeX\xspace}

% Prints the month name (e.g., January) and the year (e.g., 2008)
\newcommand{\monthyear}{%
  \ifcase\month\or January\or February\or March\or April\or May\or June\or
  July\or August\or September\or October\or November\or
  December\fi\space\number\year
}


% Prints an epigraph and speaker in sans serif, all-caps type.
\newcommand{\openepigraph}[2]{%
  %\sffamily\fontsize{14}{16}\selectfont
  \begin{fullwidth}
  \sffamily\large
  \begin{doublespace}
  \noindent\allcaps{#1}\\% epigraph
  \noindent\allcaps{#2}% author
  \end{doublespace}
  \end{fullwidth}
}

% Inserts a blank page
\newcommand{\blankpage}{\newpage\hbox{}\thispagestyle{empty}\newpage}

\usepackage{units}

% Typesets the font size, leading, and measure in the form of 10/12x26 pc.
\newcommand{\measure}[3]{#1/#2$\times$\unit[#3]{pc}}

% Macros for typesetting the documentation
\newcommand{\hlred}[1]{\textcolor{Maroon}{#1}}% prints in red
\newcommand{\hangleft}[1]{\makebox[0pt][r]{#1}}
\newcommand{\hairsp}{\hspace{1pt}}% hair space
\newcommand{\hquad}{\hskip0.5em\relax}% half quad space
\newcommand{\TODO}{\textcolor{red}{\bf TODO!}\xspace}
\newcommand{\na}{\quad--}% used in tables for N/A cells
\providecommand{\XeLaTeX}{X\lower.5ex\hbox{\kern-0.15em\reflectbox{E}}\kern-0.1em\LaTeX}
\newcommand{\tXeLaTeX}{\XeLaTeX\index{XeLaTeX@\protect\XeLaTeX}}
% \index{\texttt{\textbackslash xyz}@\hangleft{\texttt{\textbackslash}}\texttt{xyz}}
\newcommand{\tuftebs}{\symbol{'134}}% a backslash in tt type in OT1/T1
\newcommand{\doccmdnoindex}[2][]{\texttt{\tuftebs#2}}% command name -- adds backslash automatically (and doesn't add cmd to the index)
\newcommand{\doccmddef}[2][]{%
  \hlred{\texttt{\tuftebs#2}}\label{cmd:#2}%
  \ifthenelse{\isempty{#1}}%
    {% add the command to the index
      \index{#2 command@\protect\hangleft{\texttt{\tuftebs}}\texttt{#2}}% command name
    }%
    {% add the command and package to the index
      \index{#2 command@\protect\hangleft{\texttt{\tuftebs}}\texttt{#2} (\texttt{#1} package)}% command name
      \index{#1 package@\texttt{#1} package}\index{packages!#1@\texttt{#1}}% package name
    }%
}% command name -- adds backslash automaticallygit@github.com:stu314159/Nuclear_Plant_Engineering.git
\newcommand{\doccmd}[2][]{%
  \texttt{\tuftebs#2}%
  \ifthenelse{\isempty{#1}}%
    {% add the command to the index
      \index{#2 command@\protect\hangleft{\texttt{\tuftebs}}\texttt{#2}}% command name
    }%
    {% add the command and package to the index
      \index{#2 command@\protect\hangleft{\texttt{\tuftebs}}\texttt{#2} (\texttt{#1} package)}% command name
      \index{#1 package@\texttt{#1} package}\index{packages!#1@\texttt{#1}}% package name
    }%
}% command name -- adds backslash automatically
\newcommand{\docopt}[1]{\ensuremath{\langle}\textrm{\textit{#1}}\ensuremath{\rangle}}% optional command argument
\newcommand{\docarg}[1]{\textrm{\textit{#1}}}% (required) command argument
\newenvironment{docspec}{\begin{quotation}\ttfamily\parskip0pt\parindent0pt\ignorespaces}{\end{quotation}}% command specification environment
\newcommand{\docenv}[1]{\texttt{#1}\index{#1 environment@\texttt{#1} environment}\index{environments!#1@\texttt{#1}}}% environment name
\newcommand{\docenvdef}[1]{\hlred{\texttt{#1}}\label{env:#1}\index{#1 environment@\texttt{#1} environment}\index{environments!#1@\texttt{#1}}}% environment name
\newcommand{\docpkg}[1]{\texttt{#1}\index{#1 package@\texttt{#1} package}\index{packages!#1@\texttt{#1}}}% package name
\newcommand{\doccls}[1]{\texttt{#1}}% document class name
\newcommand{\docclsopt}[1]{\texttt{#1}\index{#1 class option@\texttt{#1} class option}\index{class options!#1@\texttt{#1}}}% document class option name
\newcommand{\docclsoptdef}[1]{\hlred{\texttt{#1}}\label{clsopt:#1}\index{#1 class option@\texttt{#1} class option}\index{class options!#1@\texttt{#1}}}% document class option name defined
\newcommand{\docmsg}[2]{\bigskip\begin{fullwidth}\noindent\ttfamily#1\end{fullwidth}\medskip\par\noindent#2}
\newcommand{\docfilehook}[2]{\texttt{#1}\index{file hooks!#2}\index{#1@\texttt{#1}}}
\newcommand{\doccounter}[1]{\texttt{#1}\index{#1 counter@\texttt{#1} counter}}

% Generates the index
\usepackage{makeidx}
\makeindex

\begin{document}

\begin{fullwidth}
\thispagestyle{empty}
\begin{figure}
\includegraphics[width=1.5\textwidth]{cover1.png}
\end{figure}
\end{fullwidth}

% Front matter
\frontmatter

% r.1 blank page
%\blankpage


% v.2 epigraphs
\newpage\thispagestyle{empty}
\vfill
\openepigraph{%
When in doubt, multiply both sides by an orthogonal function and integrate.
}{P.L. Chebyshev}

\vfill
\openepigraph{%
The purpose of computing is insight, not pictures
}{L.N. Trefethen}


\vfill
\openepigraph{%
Never do a calculation until you already know the answer.
}{J.A. Wheeler}


% r.3 full title page
\maketitle


% v.4 copyright page
\newpage
\begin{fullwidth}
~\vfill
\thispagestyle{empty}
\setlength{\parindent}{0pt}
\setlength{\parskip}{\baselineskip}
Copyright \copyright\ \the\year\ \thanklessauthor
\par\smallcaps{Published by \thanklesspublisher}

%\par\smallcaps{tufte-latex.github.io/tufte-latex/}
%
%\par Licensed under the Apache License, Version 2.0 (the ``License''); you may not
%use this file except in compliance with the License. You may obtain a copy
%of the License at \url{http://www.apache.org/licenses/LICENSE-2.0}. Unless
%required by applicable law or agreed to in writing, software distributed
%under the License is distributed on an \smallcaps{``AS IS'' BASIS, WITHOUT
%WARRANTIES OR CONDITIONS OF ANY KIND}, either express or implied. See the
%License for the specific language governing permissions and limitations
%under the License.\index{license}

\par\textit{First printing, \monthyear}
\end{fullwidth}

% r.5 contents
\tableofcontents

\listoffigures

\listoftables

% r.7 dedication
%\cleardoublepage
%~\vfill
%\begin{doublespace}
%\noindent\fontsize{18}{22}\selectfont\itshape
%\nohyphenation
%Dedicated to the midshipmen of the United States Naval Academy; the future of our armed services and of %our country.
%\end{doublespace}
%\vfill
%\vfill


% r.9 introduction
\cleardoublepage
\chapter*{Preface}

The purpose of this text is to provide a concise reference for engineering students who would like to strengthen their conceptual understanding and practical proficiency in analytical and numerical methods in engineering.  The material is based on a sequence of two courses taught at the United States Naval Academy.  

\subsection*{Analytical Methods}

The first course focused on analytical methods for linear ordinary and partial differential equations.  All students came into the course having taken a three-semester sequence of calculus along with a course in ordinary differential equations.  The analytical methods portion quickly reviews methods for constant coefficient linear equations and proceeds to methods for non-constant coefficients including Cauchy-Euler equations, power series methods, and method of Frobenieus.  After a review of Fourier Series methods and an introduction to Fourier-Legendre and Fourier-Bessel expansions we thoroughly explore solutions to second-order, linear, partial differential equations.  Since many students are also studying nuclear engineering, there is a heavy focus on addressing boundary value problems in cylindrical and spherical coordinate systems that are applicable to other topics of interest such as reactor physics.  There is also heavy emphasis on heat transfer applications that students will see later on in their undergraduate curriculum.

The materials presented are based heavily on Professor Dennis Zill's excellent book.\cite[-3cm]{zill2020advanced} We lightly select from chapters 1-3 for review; chapter 5 for series solution methods; and chapters 12-14 for Fourier Series and solutions to linear boundary value problems.  Material from that text is used throughout this book.  

What distinguishes this course from Prof Zill's work is the incorporation of computational tools in the solution process.  These ``semi-analytical methods'' are presented here in MATLAB\cite[-3.75cm]{matlab} owing to the students preparation with that tool.  Other open-source tools like Octave\cite[-3.5cm]{octave} and Python,\cite[-1cm]{10.5555/1593511} of course, could be used. 

\subsection*{Numerical Methods}

%%
% Start the main matter (normal chapters)
\mainmatter

\part{Introduction and Review}

\chapter{Lecture 1 - Introduction, Definitions and Terminology}
\label{ch:lec1}%<< think of better label and chapter names
\section{Objectives}
The objectives of this lecture are:
\begin{itemize}
\item Provide an overview of course content
\item Define basic terms related to differential equations
\item Provide examples of classification schemes for differential equations
\end{itemize}

\section{Course Introduction}
\newthought{This course is intended} as a one-semester introduction to partial differential equations.  It is assumed that all students have a thorough background in single- and multi-variable calculus as well as differential equations.  The first few lectures comprise a review of the portions of differential equations on which this course most heavily relies.  This is followed by a treatment of power series methods and the method of Frobenius.  These are needed so that students will understand the origins of Legendre Polynomials and Bessel functions that will be used in the solution of boundary value problems in spherical and cylindrical coordinates respectively.

\newthought{The main body} of material deals with the solution of (mostly homogeneous) boundary value problems---wave equation, heat equation, and Laplace equation---in rectangular, polar/cylindrical, and spherical coordinate systems.  For this a preparatory review of Fourier series expansions along with Fourier-Legendre and Fourier-Bessel expansions are introduced along with a leavening of Sturm-Liouville theory in boundary value problems.  The rest is a problem-by-problem tour of methods and analysis with heavy emphasis on heat transfer and nuclear engineering applications.

\section{Classification of Differential Equations}
\newthought{It is important} to be able to classify differential equations.  In this class we will learn a variety of techniques to find the function that satisfies a differential equation along with its boundary or initial conditions.\sidenote{Consider the differential equation: $\frac{du}{dx} = ux$.  The variable $u$ stands for the function, $u(x)$, that satisfies the equation; $u$ is also referred to as the \textbf{dependent variable}.  The variable $x$ is the \textbf{independent variable}.  By convention we will use the variables $x,y,z$ and $r,\theta,\phi$ as spatial independent variables and $t$ as an independent variable for time dependent problems.  We will use many other letters to denote dependent variables but most commonly $u,v,$ and $w$.} The techniques we learn in this class are tailored for specific classes of problems; you classify the problem and that tells you what method to use.  If you improperly classify the equation, you will likely use an inappropriate method and may have trouble figuring out why it is not working.  
\subsection{Classification by Type and Order}
\newthought{We shall start} with the easiest classification categories: type and order.  There are two \emph{types} of differential equations that we will consider: ordinary differential equations; and partial differential equations.  
\newthought{In an ordinary} differential equation\marginnote{\textbf{Example ODE:} $$\frac{d^2u}{dt^2}+t \frac{du}{dt}=3e^{-t}$$  There is one independent variable, $t$}, there is only one independent variable.  In a \emph{partial} differential equation,\marginnote{\textbf{Example PDE:} $$\frac{\partial^2u}{\partial x^2} + \frac{\partial^2u}{\partial y^2} = 0 $$ There are two independent variables, $x,$ and $y$.} there are multiple independent variables and consequently derivatives of the dependent variable will partial derivatives.
\newthought{The order of} a differential equation is the order of the highest derivative in the equation.  This is typically not confusing for students. If anything needs to be added here it is to be mindful of the difference between a higher order derivative and an exponent.  For example, in the second order, non-linear, ordinary differential equation shown below, 
$$ \frac{d^2u}{dx^2}+5\left(\frac{du}{dx} \right)^3 - 4u = e^{x}  $$
it isn't \emph{too} hard to realize that the ``3'' is an exponent and the ``2'' denotes a second derivative. Still, be mindful.   

\subsection{Classification by Linearity}
An $n$-th order ordinary differential equation is said to be \emph{linear} when it can be written in the form shown in Equation \ref{eq:lin_ode}:
\begin{equation}
a_n(x)u^{(n)}+a_{n-1}(x)u^{(n-1)}+\cdots+a_1(x)^{\prime}+a_0(x)u = g(x)
\label{eq:lin_ode}
\end{equation}
The key features that you should note in the form of Equation \ref{eq:lin_ode} are:
\begin{enumerate}
\item The \emph{dependent} variable and \emph{all of its derivatives} are of the first degree; that is, the power of each term involving $u$ is 1.
\item The coefficients of each term, $a_n(x)$, depend at most on the \emph{independent} variable.
\end{enumerate}
A lot of students struggle with discriminating between linear and nonlinear ODEs but it really is as simple as checking these two things.\marginnote{\textbf{Why is this important?} Most of the techniques we will learn in this course \emph{depend} upon the fact that the equation we are trying to solve is \emph{linear}.  In \emph{``the wild''} you may be presented with (or, more likely \emph{derive}) an equation and may not be explicitly told whether or not the equation is linear.  If the equation is \textbf{not} linear you will find that most of the tools you learn in this course will not be applicable; you will most likely need to use a numerical method. You need to be able to tell the difference so you know what tools to use.}  If both conditions are satisfied; the equation is linear.  If not, the equation is nonlinear.  As examples, Equation \ref{eq:nonlin_1} violates the first criterion; Equation \ref{eq:nonlin_2} violates the second.
\begin{equation}
\frac{d^2u}{dx^2}+u^2 = 0 
\label{eq:nonlin_1}
\end{equation}
\begin{equation}
\frac{d^3u}{dx^3}-5u\frac{du}{dx}=x
\label{eq:nonlin_2}
\end{equation}
\section{Verification of an Explicit Solution} 
A solution in which the dependent variable is expressed solely in terms of the independent variable and constants is said to be an \emph{explicit} solution.\marginnote{\textbf{Example explicit solution:} $u(x) = f(x)$.} Otherwise, the solution is \emph{implicit}.\marginnote{\textbf{Example implicit solution:} $G(x,u) = 0$} 

\newthought{In this class} we will mainly be interested in finding explicit solutions to differential equations that we are given or have derived.  There are some cases, however, where we are given a function and we wish to verify that it is a solution to a given differential equation.  To do this, we simply ``plug'' the equation into the differential operator and verify that an identity is derived.

\begin{example}
\textbf{Example:} Verify that $u = \frac{6}{5} - \frac{6}{5}e^{-20t}$ is a solution to:
$$\frac{du}{dt} + 20u = 24$$
\vspace{0.5cm}
\textbf{Solution:}
Since $\frac{du}{dt} = \frac{d}{dt}(\frac{6}{5} - \frac{6}{5}e^{-20t}) = 24e^{-20t}$, we can see that:
\begin{align*}
\frac{du}{dt} + 20u &= 24e^{-20t} + 20(\frac{6}{5} - \frac{6}{5}e^{-20t}) \\
&= 24e^{-20t} + 24 - 24e^{-20t} \\
&=24
\end{align*} 
which is the expected identity.
\end{example} 




\part{Power Series Methods}


%%
% The back matter contains appendices, bibliographies, indices, glossaries, etc.

\part{Back Matter}
\backmatter

\bibliography{Mathematical_Methods_for_Engineers}
\bibliographystyle{plainnat}


\appendix
\appendixpage
\noappendicestocpagenum
\addappheadtotoc



\chapter{Matlab Style Rules}
\begin{enumerate}
\item \textbf{rule:} All scripts will start with the commands: \textbf{clear}, \textbf{clc}, and \textbf{close 'all'}

\textbf{rationale:} No script should depend upon any data visible in the MATLAB workspace when the script starts.  By omitting these commands, residual data within the workspace may hide errors.

\item \textbf{rule:} Your code must be documented with enough details such that a reader unfamiliar with your work will know what you are doing.

\textbf{rationale:} Code documentation is a habit. For more significant projects readers may need help in deciding what the author of the code intended.  For your own code, the most likely reader is you---a few months into the future.

\item \textbf{rule:} Function and variable names must be meaningful and reasonable in length.

\textbf{rationale:} Failing to do either make code harder to read and maintain.

\item \textbf{rule:} All outputs from the code \underline{\textbf{must}} be meaningful; numbers should be formatted, part of a sentence, and include units. Graphs should be readable and axis labels should make sense and include units.

\textbf{rationale:} Code output is a form of communication. It is important that this communication be clear and unambiguous.

\item \textbf{rule:} Do not leave warnings from the Code Analyzer unaddressed.

\textbf{rationale:} Sometimes Code Analyzer warnings can be safely ignored.  Most of the time the warning points to a stylistic error that would be unacceptable in software that you use. Occasionally these warnings are indicative of a hidden error.

\item \textbf{rule:} Use the ``smart indentation tool'' to format the indentation of your code.

\textbf{rationale:} This tool improves code readability. It will also occasionally point out errors that you did not see before.

\item \textbf{rule:} Pre-allocate arrays; if possible initialize with \textbf{NaN} values.

\textbf{rationale:} Pre-allocation improves performance and helps readability. Initialization with \textbf{NaN} helps avoid a range of potential logical errors.

\item \textbf{rule:} Avoid ``magic numbers'' --- i.e. hard-coded constants.

\textbf{rationale:} Constants included in your code tend to hide your program logic. Also, ``magic numbers'' make code maintenance more difficult and error prone.

\item \textbf{rule:} Only write one statement per line.

\textbf{rationale:} Multi-statement-lines hurt code readability in almost all cases.

\item \textbf{rule:} Do not write excessively long lines of code; use the line continuation ``...'' and indentation to spread long expressions over several lines.

\textbf{rationale:} Following this rule improves code readability.
\end{enumerate}


\printindex
\end{document}

