\chapter{Lecture 16 - Fourier Series}
\label{ch:lec16}
\section{Objectives}
\begin{itemize}
\item Review trigonometric Series.
\item Derive/show the formulas for expansion of a function as a Fourier series.
\item Discuss periodic extensions of non-periodic functions, sine/cosine expansions, and convergence behavior.
\end{itemize}

\section{Review of Fourier Series}
In the last lecture we learned about orthogonal functions and sets of orthogonal functions.  We stated that most functions can be expressed as a linear combination of those orthogonal functions:
\begin{equation*}
u(x) = \sum\limits_{n=0}^{\infty} c_n \phi_n
\end{equation*}
where $\phi_n$ are members of a set of orthogonal functions and $c_n$ are determined by:
\begin{equation*}
c_n = \frac{(u,\phi_n)}{||\phi_n||^2}\phi_n
\end{equation*} \index{Fourier Series}
You should already have experience with expansions such as this from your previous classes in differential equations in the form of Fourier series expansions.  In this case the orthogonal functions, $\phi_n(x)$, are:
\begin{equation*}
\left\{1,\cos{\frac{\pi x}{p}},\cos{\frac{2\pi x}{p}},\dots,\sin{\frac{\pi x}{p}}, \sin{\frac{2\pi x}{p}}, \dots \right\}
\end{equation*}
\marginnote[-1.5cm]{The function $\phi(x) = 1$ could also be written: $\cos{\frac{0 \pi x}{p}}$.}where $p$ indicates the \emph{period}.\sidenote[][-0.5cm]{Reminder that a function, $f(x)$, is periodic with period $p$ if $f(x+p) = f(x)$.} 

That members of this set of functions are all mutually orthogonal can be directly shown; we will demonstrate this for the members of the form $\phi_n(x) = \cos{\sfrac{n \pi x}{p}}$; other cases are left for homework exercises.

\vspace{1.0cm}

\noindent\textbf{Example:} Show that functions of the form $\phi_n(x) = \cos{\frac{n\pi x}{p}}$ are orthogonal over the interval $x \in [-p,p]$:

\vspace{0.5cm}
\noindent Consider two functions, $\phi_n(x)$ and $\phi_m(x)$ where $m,n$ are integers and $m \ne n$.  The functions are orthogonal on the interval $x\in [-p,p]$ if $(\phi_n,\phi_m) = 0$.  From the definition of the inner product:\marginnote[0.5cm]{Here we use the identity: $\cos{(\alpha \pm \beta)} = \cos{\alpha}\cos{\beta} \mp \sin{\alpha}\sin{\beta}$. So that
\begin{align*}
\cos{\left(\alpha + \beta\right)} &= \cos{\alpha}\cos{\beta} - \sin{\alpha}\sin{\beta} \\
+\cos{\left(\alpha - \beta\right)} &= \cos{\alpha}\cos{\beta} + \sin{\alpha}\sin{\beta} \\
&= 2\cos{\alpha}\cos{\beta}
\end{align*}
and therefore
\begin{align*}
\cos{\alpha}\cos{\beta} &= \frac{1}{2}\left[\cos{\left(\alpha + \beta\right)} + \cos{\left(\alpha - \beta\right)}\right]
\end{align*}
}
\begin{align*}
(\phi_n,\phi_m) &= \int_{-p}^{p} \cos{\frac{n \pi x}{p}} \cos{\frac{m \pi x}{p}} \ dx \\
&= \frac{1}{2}\int_{-p}^{p} \cos{(n+m)\frac{\pi x}{p}} + \cos{(n-m)\frac{\pi x}{p}} \ dx \\
&= \frac{1}{2} \left[\frac{1}{n+m} \frac{p}{\pi}\sin{(n+m)\frac{\pi x}{p}}\Bigl|_{-p}^{p} + \frac{1}{n-m}\frac{p}{\pi}\sin{(n-m)\frac{\pi x}{p}}\Bigl|_{-p}^{p} \right] \\
&=0
\end{align*}
where the last terms are zero since we are evaluating the sine function at integer multiples of $\pi$.  This shows, at least, all of the cosine members are orthogonal.  For the case $m=n$ we get:\marginnote[1.0cm]{Rather than derive this rigorously, we will combine a tabulated result of standard integrals: $\int \cos^2 u \ du = \frac{1}{2}u + \frac{1}{4}\sin{2u}+C$, with $u$ substitution.}
\begin{align*}
\left(\phi_n,\phi_n \right) &= \int_{-p}^{p} \cos{\left(\frac{n \pi x}{p} \right)}^2 \\
&= \frac{p}{n \pi} \left[\frac{1}{2}\frac{n \pi x}{p} + \frac{1}{4} \sin{\frac{2 n \pi x}{p}} \right]\Bigl|_{-p}^{p} \\
&= \frac{p}{2} - \frac{-p}{2} \\
&= p
\end{align*}

\newthought{We can use} this infinite set of orthogonal functions to represent any other continuous function over the interval $[-p,p]$.\marginnote{You might be wondering at this point why you would ever want to represent a function $f(x)$ as a linear combination of orthogonal functions.  The answer is that the members of the set of orthogonal functions are solutions to a linear homogeneous boundary value problem and the function $f(x)$ will be a boundary condition for a partial differential equation that we are trying to solve.}  In your differential equations class you were taught to do this by using Equation \ref{eq:fourier-exp}:

\begin{equation}
f(x) = \frac{a_0}{2} + \sum\limits_{n=1}^{\infty}\left[a_n \cos{\frac{n \pi x}{p}} + b_n \sin{\frac{n \pi x}{p}} \right]
\label{eq:fourier-exp}
\end{equation}
We can solve for the coefficients $a_0$, $a_n$ and $b_n$ one at a time by multiplying both sides of Equation \ref{eq:fourier-exp} by the corresponding orthogonal function, and integrating.\sidenote{In more formal mathematical terms: we take the \emph{inner product} of both sides with respect to the orthogonal function, but of course that means the same thing.}  The orthogonal function corresponding to $a_0$ is 1; so to find $a_0$ we multiply both sides of Equation \ref{eq:fourier-exp} by 1 and integrate:
\begin{align*}
f(x) &= \frac{a_0}{2}+\sum\limits_{n=1}^{\infty} \left[a_n \cos{\frac{n \pi x}{p}} + b_n \sin{\frac{n \pi x}{p}} \right] \\
\int_{-p}^{p}f(x)(1) \ dx &= \int_{-p}^{p}\frac{a_0}{2}(1) \ dx + \underbrace{\int_{-p}^{p} \sum\limits_{n=1}^{\infty} \left[a_n \cos{\frac{n \pi x}{p}} + b_n \sin{\frac{n \pi x}{p}} \right](1) \ dx}_{= 0 \text{ due to orthogonality}} \\
\int_{-p}^{p} f(x) \ dx &= \frac{a_0}{2} 2p + 0 \\
\Rightarrow a_0 &= \frac{1}{p}\int_{-p}^{p}f(x) \ dx
\end{align*}
To get the value of $a_1$, we multiply both sides by $\cos{\sfrac{\pi x}{p}}$ and integrate:
\begin{fullwidth}
\begin{multline*}
\int_{-p}^{p} f(x)\cos{\frac{\pi x}{p}} \ dx = \frac{a_0}{2} \cancelto{0}{\int_{-p}^{p}(1)\cos{\frac{\pi x}{p}} \ dx} + \cdots \\
a_1 \underbrace{\int_{-p}^{p}\cos{\frac{\pi x}{p}}^2 \ dx}_{=p} + b_1 \cancelto{0}{\int_{-p}^{p} \sin{\frac{\pi x}{p}}\cos{\frac{\pi x}{p}} \ dx} + a_2 \cancelto{0}{\int_{-p}^{p}\cos{\frac{2\pi x}{p}}\cos{\frac{\pi x}{p}} \ dx}  + \cdots
\end{multline*}
\end{fullwidth}
Solving for $a_1$ we get:
\begin{align*}
&\int_{-p}^{p} f(x)\cos{\frac{\pi x}{p}} \ dx =a_1 p \\
& \Rightarrow a_1 = \frac{1}{p}\int_{-p}^{p} f(x)\cos{\frac{\pi x}{p}} \ dx
\end{align*}
We repeat the process for $b_1$ by multiplying both sides by $\sin{\sfrac{\pi x}{p}}$; for $a_n$ we use $\cos{\sfrac{n \pi x}{p}}$ and for $b_n$, $\sin{\sfrac{n \pi x}{p}}$.  The resulting formulas for the coefficients are given in Equation \ref{eq:Fourier-Coeff}.
\begin{align}
\begin{split}
a_0 &= \frac{1}{p}\int_{-p}^{p} f(x) \ dx \\
a_n &= \frac{1}{p}\int_{-p}^{p} f(x) \cos{\frac{n \pi x}{p}} \ dx \\
b_n &= \frac{1}{p}\int_{-p}^{p} f(x) \sin{\frac{n \pi x}{p}} \ dx 
\end{split}
\label{eq:Fourier-Coeff}
\end{align}
Since this is an infinite series, we need to concern ourselves with convergence.  The theorem below provides us assurance of convergence for continuous and piece-wise continuous functions on the interval $[-p,p]$.  \marginnote[1.0cm]{In coming lectures and when doing assignments you will see that issues of continuity of $f$ and $\sfrac{df}{dx}$ have obvious visible influence on the convergence behavior of Fourier Series.}
\begin{theorem}{Convergence of Fourier Series}
If $f$ and $\sfrac{df}{dx}$ are piece-wise continuous on an interval $[-p,p]$ then for all $x$ in the interval $[-p,p]$ the Fourier Series converges to $f$ at points where the function is continuous; at points of discontinuity, the Fourier series converges to 
\begin{equation*}
\frac{f(x^{-}) + f(x^+)}{2}
\end{equation*}
where $f(x^-)$ and $f(x^+)$ denote the limit of $f(x)$ from the left and right at the point of discontinuity.
\end{theorem}
In the next few lectures we will define other orthogonal function expansions similar to the Fourier Series.  Nonetheless, for periodic functions defined on a finite interval, the Fourier series provides the best representation of a function.\marginnote{When we say ``best'' representation, we (more or less) mean two things:
\begin{enumerate}
\item $|| \tilde{f}_n - f||$, where $\tilde{f}_n$ is the power series representation of $f$ up to $n$ terms, gets smaller with fewer terms than other series expansions; and 
\item calculation of the coefficients $a_n$ and $b_n$ can be carried out with greater numeric stability than for other expansions.
\end{enumerate}
}  There are some special cases, however, where we can take advantage of structural properties of $f(x)$ to reduce the amount of work we need to do in carrying out the Fourier series expansions.

\subsection{Even Functions and Odd Functions} \index{even function} \index{odd function}
When doing a Fourier series expansion it is sometimes helpful to consider whether a function is \emph{even} or \emph{odd}.  
\begin{marginfigure}
\includegraphics{lec16_even.png}
\caption{An example even function.}
\label{fig:even-fun}
\end{marginfigure}
\begin{marginfigure}
\includegraphics{lec16_odd.png}
\caption{An example odd function.}
\label{fig:odd-fun}
\end{marginfigure}
\begin{definition}[Even Function]
A function is even if, for all real values $x$, $f(-x) = f(x)$.  
\end{definition}
An example of an even function is shown in Figure \ref{fig:even-fun}.
\begin{definition}[Odd Function]
A function is even if, for all real values $x$, $f(-x) = -f(x)$.
\end{definition}
An example of an odd function is shown in Figure \ref{fig:odd-fun}.

\newthought{Some properties} of even an odd functions include:\sidenote{Students are welcome to prove these assertions.}
\begin{enumerate}
\item an even function times an even function results in an even function;
\item an odd function times an odd function results in an even function;
\item an even function times an odd function results in an odd function;
\item adding or subtracting two even functions results in an even function;
\item adding or subtracting two odd functions results in an odd function;
\item $\int_{-p}^{p} f_{\text{even}}(x) \ dx = 2\int_{0}^{p} f_{\text{even}}(x) \ dx$
\item $\int_{-p}^{p} f_{\text{odd}}(x) \ dx = 0$.
\end{enumerate}

The ``even-ness'' or ``odd-ness'' of a function is relevant to Fourier series expansions.  If you expand an \emph{even} function in a Fourier series you will find that all of the $b_n$ coefficients are zero;  if you expand an \emph{odd} function in a Fourier series you will find that $a_0$ and $a_n$ terms are all zero.

You can still use the formulas presented in Equation \ref{eq:Fourier-Coeff} when expanding even or odd functions.  Alternatively, you can use the formulas for the Cosine expansion or Sine expansion below for even or odd functions respectively.

\vspace{0.5cm}

\noindent\textbf{Cosine series:}
\begin{align}
\begin{split}
f(x) &= \frac{a_0}{2} + \sum\limits_{n=1}^{\infty}a_n \cos{\frac{n \pi x}{p}} \\
a_0 &= \frac{2}{p}\int_0^p \ f(x) \ dx \\
a_n &= \frac{2}{p} \int_0^p \ f(x) \cos{\frac{n \pi x}{p}} \ dx
\end{split}
\end{align}

\vspace{0.5cm}

\noindent\textbf{Sine series:}
\begin{align}
\begin{split}
f(x) &= \sum\limits_{n=1}^{\infty} b_n \sin{\frac{n \pi x}{p}} \\
b_n &= \frac{2}{p}\int_{0}^{p} \ f(x) \ \sin{\frac{n \pi x}{p}} \ dx
\end{split}
\end{align}

