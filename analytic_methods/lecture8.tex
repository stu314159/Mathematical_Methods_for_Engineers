\chapter{Lecture 8 - Power Series Solutions at Ordinary Points}
\label{ch:lec8}
\section{Objectives}
The objectives of this lecture are:
\begin{itemize}
\item Introduce some definitions and concepts relevant for power series solutions of differential equations.
\item Do some example problems.
\end{itemize}

\section{Introduction}
In this section we will restrict our attention to second-order, linear, homogeneous differential equations in standard form as shown in Equation \ref{eq:SOLH-std}.

\begin{equation}
u^{\prime \prime}+P(x)u^{\prime}+Q(x)u=0
\label{eq:SOLH-std}
\end{equation}

\index{point, ordinary} \index{point, singular}
\begin{definition}[Ordinary Points and Singular Points]
A point $x_0$ is said to be an \emph{ordinary point} of a differential equation if both $P(x)$ and $Q(x)$ in the standard form are \emph{analytic} at $x_0$.  A point that is not an \emph{ordinary point} is a \emph{singular point}.
\end{definition}


\begin{theorem}[Existence of Power Series Solutions]
If $x=x_0$ is an \emph{ordinary point} of the differential equation, we can always find two linearly independent solutions in the form of a power series centered at $x_0$.  A series solution converges at least on some interval defined by $\left| x - x_0 \right|<R$ where $R$ is the distance from $x_0$ to the closest singular point.
\label{thm:existence-of-power-series-solutions}
\end{theorem}
\marginnote[-1.5cm]{Let us emphasize: this theorem applies only to second-order, linear, homogeneous differential equations.}

The basic strategy we will use to find power series solutions for linear differential equations with variable coefficients where $P(x)$ and $Q(x)$ are analytic in the domain of interest is:

\begin{enumerate}
\item Find solutions in the form of a power series by substituting $u = \sum\limits_{n=0}^{\infty} c_n x^n$ into the differential equation.
\item Solve for the values of the coefficients by equating the coefficients on the left with those on the right (e.g. zero for homogeneous equations); and
\item The equations (often 2- or 3-term recurrence relations) for the series coefficients \emph{defines} the function that is the solution of the differential equation.\marginnote[-1.05cm]{\textbf{Important:} the series ``solution'' is only valid if the series so-derived has a non-zero radius of convergence.}
\end{enumerate}

\section{Examples}

\noindent\textbf{Example:} Solve $u^{\prime \prime}+u = 0$ using the power series method; compare with the known solution $u(x)=c_0\cos{x}+c_1\sin{x}$\marginnote[-1.5cm]{It is not \emph{required} that a problem have variable coefficients in order to use the Power Series method; only that $P(x)$ and $Q(x)$ are analytic on the domain of interest.  Constants are always analytic over the entire real number line so you can always use the Power Series method on linear, constant-coefficient differential equations.}

\newthought{In accordance with} our strategy, we will assume that the solution is of the form: $u(x)=\sum_{n=0}^{\infty} c_n x^n$.  This means that $u^{\prime} = \sum_{n=1}^{\infty} n c_n x^{n-1}$ and $u^{\prime \prime} = \sum_{n=2}^{\infty} n(n-1)c_nx^{n-2}$.\marginnote{Note that the $n=0$ term in $u^{\prime}$ is omitted as is the $n=0$ and $n=1$ term in $u^{\prime \prime}$.  These terms are zero due to having taken the first- and second-derivative on the constant $(x^0)$ and linear ($x^1)$ terms of the power series.}  Plugging this into our differential equation gives us:

\begin{align*}
u^{\prime \prime} + u &= 0 \\
\sum\limits_{n=2}^{\infty} n(n-1)c_nx^{n-2} + \sum\limits_{n=0}^{\infty} c_n x^n &= 0
\end{align*}

\noindent We need to combine the two summations.  It is clear that the summation indexes start at different values but we are lucky in that the summations are already ``in phase'' since the first term in each summation is a constant $(x^0)$ term.

\begin{equation*}
\underbrace{\sum\limits_{n=2}^{\infty} n(n-1)c_nx^{n-2}}_{\substack{k=n-2 \\ n=k+2}} + \underbrace{\sum\limits_{n=0}^{\infty} c_n x^n}_{\substack{k=n \\ n=k}} = 0
\end{equation*}

\noindent For the first summation $k=n-2$, so $n=k+2$; we will use these definitions to re-write the first summation.  For the second summation, $k=n$ so all we need to do for the second summation is replace all the $n$'s with $k$'s.  The results of these substitutions and the combined summation are shown below:

\begin{align*}
\sum\limits_{k=0}^{\infty} (k+2)(k+1)c_{k+2}x^k + \sum\limits_{k=0}^{\infty} c_k x^k &= 0 \\
\sum\limits_{k=0}^{\infty}\underbrace{\left[(k+2)(k+1)c_{k+2} + c_k \right]}_{\text{coefficients for new power series}}x^k&=0
\end{align*}

\newthought{The expression} $\left[(k+2)(k+1)c_{k+2}+c_k \right]$ is now a formula for the coefficients of a new power series. This power series, according to the equation, is equal to \emph{zero} so that means, per Definition \ref{def:ps-identity-property}, all of the coefficients must be equal to zero:

\marginnote[1.0cm]{This is called a \emph{two-term recurrence} since the expression involves \emph{two} terms; $c_{k-2}$ and $c_k$.}
\begin{equation*}
(k+2)(k+1)c_{k+2} + c_{k} = 0, \ \ \text{for all } k \in [0,2,3,\dots]
\end{equation*} 


By convention, we will re-write this recurrence relation to solve for the \emph{higher}-index coefficients in terms of the \emph{lower}-index coefficients.  We do this in Equation \ref{eq:ex1-l8-rec}:

\begin{equation}
c_{k+2}=-\frac{c_k}{(k+2)(k+1)}
\label{eq:ex1-l8-rec}
\end{equation}

\newthought{As we should} expect, there are two unknown constants in this general solution---$c_0$ and $c_1$.\marginnote[-1.0cm]{As we would expect the general solution for any other second order differential equation would have two unknown constants that can only be resolved by adding initial- or boundary-conditions.}  The first value of $k$ from the summation in our solution is $k=0$ which gives us an expression for $c_2$ in terms of $c_0$.  The second value, $k=1$, will give us an expression for $c_3$ in terms of $c_1$.  Simplified equations for the first few coefficients are presented in the table below.
\begin{table*}
\begin{tabular}{l | l | l}
$k=0$ & $k=2$ & $k=4$ \\
$c_2=\frac{-c_0}{(1)(2)}$ & $c_4=\frac{-c_2}{(3)(4)}=\frac{c_0}{4!}$ & $c_6 = \frac{-c_4}{(5)(6)} = \frac{-c_0}{6!}$ \\\hline
$k=1$ & $k=3$ & $k=5$ \\
$c_3 = \frac{-c_1}{(2)(3)}$ & $c_5 = \frac{-c_3}{(4)(5)}=\frac{c_1}{5!}$ & $c_7=\frac{-c_5}{(6)(7)} = \frac{-c_1}{7!}$\\
\end{tabular}
\end{table*}
Each cell in the table above is a formula for the $k$\textsuperscript{th}-coefficient of our power series solution.  Organizing this into a formula for our power series solution gives us:\marginnote{Notice how the even-numbered coefficients are all dependent on $c_0$ and all of the odd-numbered coefficients are dependent on $c_1$.}
\begin{align*}
u(x) &= c_0 + c_1x + c_2x^2+c_3x^3+c_4x^4+c_5x^5+c_6x^6+c_7x^7 + \cdots \\
u(x) &= c_0\left(1 + \frac{c_2}{c_0}x^2 + \frac{c_4}{c_0}x^4 + \frac{c_6}{c_0}x^6 + \cdots \right) + c_1\left(x + \frac{c_3}{c_1}x^3 + \frac{c_5}{c_1}x^5 + \frac{c_7}{c_1}x^7 + \cdots \right) \\
u(x)&=c_0\left(1-\frac{x^2}{2!}+\frac{x^4}{4!}-\frac{x^6}{6!} + \cdots  \right) + c_1\left(x -\frac{x^3}{3!}+\frac{x^5}{5!} - \frac{x^7}{7!} + \cdots \right)
\end{align*}
where in the last line we have substituted the formulas for coefficients $c_2$ through $c_7$ in terms of $c_0$ and $c_1$.

Recalling from the last lecture the power series representations of $\cos{x}$ and $\sin{x}$ and we should be able to see them again here.\marginnote{In general you are not expected to, nor will you be able to, identify common functions from a power series solution. This is a special case.}
\begin{align*}
u(x)&=c_0\underbrace{\left(1-\frac{x^2}{2!}+\frac{x^4}{4!}-\frac{x^6}{6!} + \cdots  \right)}_{\cos{x}} + c_1\underbrace{\left(x -\frac{x^3}{3!}+\frac{x^5}{5!} - \frac{x^7}{7!} + \cdots \right)}_{\sin{x}} \\
u(x)&=c_0\cos{x} + c_1\sin{x}
\end{align*}
Which is exactly what we would have determined using our methods for constant coefficient linear equations.

\vspace{3.0cm}
\noindent\textbf{Example:} Find the general solution to $u^{\prime \prime}-xu = 0$.

Notice first that while this equation is linear and homogeneous it is not constant-coefficient.  It is also not a Cauchy-Euler equation.\marginnote{The equation is separable but, in this case, the solution is not so easy to obtain using that method either.}  We will use the power series method to solve this problem.  Assuming $u=\sum_{n=0}^{\infty}c_n x^n$ and inserting this into the governing equation gives us:
\begin{align*}
u^{\prime \prime}-xu &= 0 \\
\sum\limits_{n=2}^{\infty}n(n-1)c_n x^{n-2} - x\sum\limits_{n=0}^{\infty}c_nx^n &= 0 \\
\sum\limits_{n=2}^{\infty}n(n-1)c_n x^{n-2} - \sum\limits_{n=0}^{\infty}c_n x^{n+1} &=0
\end{align*}
We want to combine these summations and see that they are both ``out of phase'' and the summation index, $n$, starts at different values for each summation.
\begin{align*}
\underbrace{\sum\limits_{n=2}^{\infty}n(n-1)c_n x^{n-2}}_{\text{for }n=2, \ \ x^0} - \underbrace{\sum\limits_{n=0}^{\infty}c_n x^{n+1}}_{\text{for }n=0, \ \ x^1} &=0 \\
\end{align*}
So we must separate out the first term in the first summation to get the summations in phase.
\begin{equation*}
(2)(1)c_2x^0 + \sum\limits_{n=3}^{\infty}n(n-1)c_n x^{n-2} - \sum\limits_{n=0}^{\infty}c_n x^{n+1} = 0
\end{equation*}
Next we must combine our indices using $k=n-2$ for the first summation and $k=n+1$ for the second summation.\marginnote[-0.5cm]{\textbf{Reminder:} we take our definition of $k$ from the \underline{exponent for $x$} in each summation term.}  
\begin{equation*}
2c_2x^0 + \underbrace{\sum\limits_{n=3}^{\infty}n(n-1)c_n x^{n-2}}_{\substack{k=n-2 \\n=k+2}} - \underbrace{\sum\limits_{n=0}^{\infty}c_n x^{n+1}}_{\substack{k=n+1 \\ n=k-1}} = 0
\end{equation*}
Doing this gives us:\marginnote[0.5cm]{Do not forget the minus sign in front of the second summation.  It is easy to miss.}
\begin{equation*}
2c_2 + \sum\limits_{k=1}^{\infty} \left[(k+2)(k+1)c_{k+2} - c_{k-1}\right]x^k = 0
\end{equation*}
In order to solve the differential equation, the coefficient for \underline{every power of $x$} needs to be zero.  To do this:
\begin{equation*}
\underbracket{2c_2}_{\Rightarrow c_2=0} + \sum\limits_{k=1}^{\infty} [\underbrace{(k+2)(k+1)c_{k+2} - c_{k-1}}_{\text{must equal zero}}]x^k = 0
\end{equation*}

\vspace{4.0cm}

\noindent Our corresponding, two-term recurrence relation is:\marginnote{\textbf{Reminder:} we should define our recurrence relation to give higher-order coefficients in terms of lower-order coefficients.}
\begin{equation*}
c_{k+2}=\frac{c_{k-1}}{(k+2)(k+1)}
\end{equation*}
We expect two arbitrary constants, $c_0$ and $c_1$ and we know from the work above that $c_2=0$ so we will start solving for constants starting with $k=1$:
\begin{table}[h!]
\begin{tabular}{l | l }
$k=1$ & $k=4$ \\
$c_3 = \frac{c_0}{(3)(2)}$ & $c_6 = \frac{c_3}{(6)(5)} = \frac{c_0}{(2)(3)(5)(6)}$ \\\hline 
$k=2$ & $k=5$ \\
$c_4 = \frac{c_1}{(3)(4)}$ & $c_7 = \frac{c_4}{(6)(7)} = \frac{c_1}{(3)(4)(6)(7)}$ \\\hline
$k=3$ & $k=6$ \\
$c_5 = \frac{c_2}{(5)(4)} = 0$ & $c_8 = \frac{c_{5}}{(8)(7)} = 0$ \\\hline
$k=7$ & $k=8$ \\
$c_9 = \frac{c_6}{(9)(8)} = \frac{c_0}{(2)(3)(5)(6)(8)(9)}$ & $c_{10}=\frac{c_7}{(10)(9)} = \frac{c_1}{(3)(4)(6)(7)(9)(10)}$ \\
\end{tabular}
\end{table}

\vspace{0.25cm}

\noindent Organizing the coefficients from the table into an equation we get:



\begin{fullwidth}
\begin{align*}
u(x) &= c_0 + c_1x + c_2x^2 + c_3x^3 + c_4x^4 + c_5x^5 + c_6x^6 + c_7x^7+c_8x^8 + c_9x^9 + c_{10}x^{10} + \cdots \\
u(x) &= c_0\left(1 + \frac{c_3}{c_0}x^3 + \frac{c_6}{c_0}x^6 + \frac{c_9}{c_0}x^9 + \cdots \right) + c_1\left(x + \frac{c_4}{c_1}x^4 + \frac{c_7}{c_1}x^7 + \frac{c_{10}}{c_1}x^{10}+\cdots \right)
\end{align*}
which can be written:
\begin{multline*}
u(x) = c_0\left(1 + \frac{x^3}{(2)(3)} + \frac{x^6}{(2)(3)(5)(6)} + \frac{x^9}{(2)(3)(5)(6)(8)(9)} + \cdots \right) + \\
c_1\left(x + \frac{x^4}{(3)(4)} + \frac{x^7}{(3)(4)(6)(7)}+\frac{x^{10}}{(3)(4)(6)(7)(9)(10)} + \cdots \right)
\end{multline*}
\end{fullwidth}
The equation we solved is known as Airy's Equation.  The power series solution is not pretty, but is a perfectly adequate representation of the function provided that we have the wherewithal to evaluate the function for a reasonable number of terms.
