\chapter{Lecture 2 - Separable and Linear 1st order Equations}
\label{ch:lec2}
\section{Objectives}
The objectives of this lecture are:
\begin{itemize}
\item Define and describe the solution procedure for \emph{separable} first order equations
\item Define and demonstrate the solution procedure for \emph{linear} first order equations
\end{itemize}

\section{Separable Equations}
A first order differential equation of the form shown below
\begin{equation}
\frac{du}{dx} = g(u)h(x)
\label{eq:separable}
\end{equation}
is said to be \emph{separable} or have \emph{separable variables.}\marginnote[-1.5cm]{\textbf{Note:} there is \textbf{no} requirement that the 1st order equation be \emph{linear}.  This is one of the few techniques that we will study in this course that can be applied to nonlinear equations.} 

\newthought{The solution method} for separable equations is, in principle simple.  For the separable differential equation given in Equation \ref{eq:separable} we would separate and integrate:

\begin{align*}
\frac{du}{dx} &= g(u)h(x) \\
\frac{du}{g(u)} &= h(x) dx \\
\int \frac{1}{g(u)} \ du &= \int h(x) \ dx 
\end{align*}
\marginnote[-3.5cm]{\textbf{Note:} there are at least two complications here.
\begin{enumerate}
\item The solution you thus derive may be either implicit or explicit.  An implicit solution is, as a practical matter, fairly inconvenient to deal with; and
\item It may not be possible to actually carry out the integrals analytically. 
\end{enumerate}
Nonetheless, we shall carry on and give it a try anyway.  
}
Generally speaking, one of your first checks for a first order equation should be: is it separable?  If so, you should separate the variables and solve.  The examples below are intended to illustrate the method.  Note that in the final example, the integral cannot be done analytically.
\begin{example}[h!]
Solve the following separable, first order differential equations
.
\textbf{Example 1:} 
\begin{align*}
\frac{du}{dx} &= \frac{u}{1 + x} \\
\frac{du}{u} &= \frac{dx}{1+x} \\
\int \frac{d}{u} &= \int \frac{dx}{1+x} \\
\ln{|u|} + c_1 &= \ln{|1+x|} + c_2 \\
|u| &= e^{\left[\ln{|1+x|} + c_3\right]}\\
    u(x)&= c|1+x|
\end{align*}

\end{example}

\begin{example}[h!]
\textbf{Example 2:}
\begin{align*}
\frac{du}{dx} &= -\frac{x}{u} \\
\int u \ du &= -\int x \ dx \\
\frac{u^2}{2} &= -\frac{x^2}{2}+c \\
u(x) &= \sqrt{c - x^2}
\end{align*}
\end{example}

\begin{example}[h!]
\textbf{Example 3:}
Solve the first order initial value problem shown below:
\begin{equation*}
\frac{du}{dx} = e^{-x^2}, \ \ u(2) = 6, \ \ 2 \le x < \infty
\end{equation*}
\begin{align*}
du &= e^{-x^2}\ dx \\
\int_2^{x} \frac{du}{dt} \ dt &= \int_{2}^{x} e^{-t^2} \ dt \\
u(x) - u(2) &= \int_{2}^{x} e^{-t^2} \ dt \\
u(x) &= 6 + \int_{2}^{x} e^{-t^2} \ dt
\end{align*}
where we have used the dummy variable $t$ in the integrals; the last integral will need to be evaluated numerically.

\end{example}

\section{Linear Equations}
A first-order differential equation of the form:
\begin{equation}
a_1(x)\frac{du}{dx} + a_0(x)u = g(x)
\label{eq:lin_first_order}
\end{equation}
is said to be a first order \emph{linear equation} in the dependent variable $u$.\marginnote[-2.0cm]{\textbf{Note:} it is sometimes customary to write the differential equation in \emph{operator form} where the differential operator, $\mathcal{L}=a_1(x) \frac{d}{dx} + a_0(x)$, is applied to the function $u(x)$ to get $g(x)$; $\mathcal{L}u(x) = g(x)$ }
When $g(x) = 0$, the first-order linear equation is said to be \emph{homogeneous}; otherwise it is \emph{non-homogeneous}.\marginnote{Notice that $g(x)$ is the only term in Equation \ref{eq:lin_first_order} that does \underline{\textbf{not}} include $u$ or any of its derivatives.}

\newthought{When solving} equations of this type it is useful to express it in the \textbf{standard form}:
\begin{equation}
\frac{du}{dx}+P(x)u = f(x)
\label{eq:first-order-linear-standard-form}
\end{equation}  
The method for solving this equation makes use of the \underline{\emph{linearity}} property\marginnote{When we say an operator is \textbf{linear}, what we mean is that the following relationships must hold:
\begin{enumerate}
\item $\mathcal{L}(\alpha u) = \alpha \mathcal{L}(u)$
\item $\mathcal{L}(u + v) = \mathcal{L}(u) + \mathcal{L}(v)$
\end{enumerate}
for functions $u$,$v$ and scalar constant $\alpha$.  Think of this as a \emph{definition} of linearity.} and express the solution in the following way: $u(x) = u_c(x) + u_p(x)$; plugging this into Equation \ref{eq:first-order-linear-standard-form} gives us:

\begin{multline}
\frac{d}{dx}[u_c + u_p] + P(x)[u_c + u_p] = \\
\left[\frac{du_c}{dx}+P(x)u_c \right] + \left[\frac{du_p}{dx}+P(x)u_p \right] = f(x) 
\label{eq:lin-first-order1}
\end{multline}
where $u_c(x)$ is the solution to the \emph{associated homogeneous problem}\marginnote{The linear operator here is: $\mathcal{L}=\frac{d}{dx} + P(x)$.  Equation \ref{eq:fol_complementary} says $\mathcal{L}u_c = 0$; Equation \ref{eq:fol_particular} says $\mathcal{L}u_p = f(x)$; Equation \ref{eq:lin-first-order1} says that $\mathcal{L}(u_c+u_p) = 0+f(x) = f(x)$.}
\marginnote[0.25cm]{What might trouble you now is: if we have $u_p$, is this not a solution to Equation \ref{eq:first-order-linear-standard-form}?  Why do we need $u_c$?  The next thing that should trouble you is that if $u_p$ is a solution, by the linearity property of $\mathcal{L}$, so is $u_p$ plus \emph{any} constant multiple of $u_c$.  The solution is not \emph{unique}}
\marginnote[0.25cm]{This will all be resolved when we recall that $u_c$ will have an arbitrary constant through which we will be able to say that $u = u_c + u_p$ is a function describing \emph{all} possible solutions of Equation \ref{eq:first-order-linear-standard-form} and the arbitrary constant in $u_c$ will be set so as to uniquely satisfy a given initial/boundary condition.}
\begin{equation}
\frac{du_c}{dx}+P(x)u_c = 0
\label{eq:fol_complementary}
\end{equation}
and $u_p(x)$ is the solution to:
\begin{equation}
\frac{du_p}{dx}+P(x)u_p = f(x)
\label{eq:fol_particular}
\end{equation}
We can see that Equation \ref{eq:fol_complementary} is separable:
\begin{align*}
\frac{du_c}{dx}+P(x)u_c &= 0 \\
\frac{du_c}{u_c} &= -P(x) \ dx \\
\ln{u_c} + C &= -\int_{P(x) \ dx} \\
u_c(x) &= e^{-\int P(x) \ dx + C_1} \\
u_c(x) &= e^{-\int P(x) \ dx}e^{C_1}
u_c(x) &=Ce^{-\int P(x) \ dx}
\end{align*}
where $C = e^{C_1}$.\marginnote{\textbf{Note:} At some point in time, I will desist in making such piddling distinctions between constants.  $C_1$ is an arbitrary constant, $e^{C_1}$ is still an arbitrary constant; there is no real difference between $C_1$ and $C$ and, in this author's humble opinion, they do not rate different symbols.}

\newthought{We need to find} a solution $u_p(x)$ to Equation \ref{eq:fol_particular}.  The technique we will use is called \emph{variation of parameters}.  It consists of looking for a solution in the form $y_p(x) = v(x)u_1(x)$, where $u_1(x) = e^{-\int P(x) \ dx}$ which is $u_c(x)$ with the arbitrary constant set to 1 and $v(x)$ might be thought of as some kind of weighting or \emph{variational} function.

\newthought{We will insert} this proposed form of $y_p(x)$ into Equation \ref{eq:fol_particular}:
\begin{equation*}
\frac{d(vu_1)}{dx} + P(x)(v(x)u_1(x))=f(x)
\end{equation*}
We apply the product rule to the first term and re-arrange terms:
\begin{align*}
u_1(x)\frac{dv}{dx}+ v(x)\frac{du_1}{dx} + P(x)(v(x)u_1(x)) &= f(x) \\
v(x)\underbrace{\left[\frac{du_1}{dx}+P(x)u_1(x) \right]}_{\textbf{= 0}}+u_1(x)\frac{dv}{dx} &= f(x) \\
u_1(x)\frac{dv}{dx} &= f(x)
\end{align*}
In the last line we can observe that the equation is \emph{separable} and thus solve:
\begin{align*}
v(x) &= \int \frac{f(x)}{u_1(x)} \ dx \\
     &= \int e^{\int P(x) \ dx}f(x) \ dx
\end{align*}
Now that we know what $v(x)$ must be, we can combine this with $u_1(x)$ to get $u_p(x)$:
\begin{equation}
u_p(x) = e^{-\int P(x) \ dx}\left[\int e^{\int P(x) \ dx}f(x) \ dx \right]
\label{eq:fol-particular-sol}
\end{equation}
Equation \ref{eq:fol-particular-sol} is messy and perhaps a bit scary but given definitions of $P(x)$ and $f(x)$ we might hope we can solve it anyway.  We now have expressions for both $u_c$ and $u_p$; they can be combined into the solution for the first-order linear equation:
\begin{equation}
u(x) = Ce^{-\int P(x) \ dx} + e^{-\int P(x) \ dx} \left[e^{\int P(x) \ dx} f(x) \ dx \right]
\label{eq:fol-solution}
\end{equation}

\section{Method of Solution}
Once we have identified a problem to be first-order and linear, we will solve the problem using the following steps:
\begin{enumerate}
\item Write the equation in standard form (Equation \ref{eq:first-order-linear-standard-form})
\item Determine the integrating factor $\mu = e^{-\int P(x) \ dx}$.
\item Solve for the general solution $u(x)$ using Equation \ref{eq:fol-solution}.
\item Apply initial/boundary condition if given.
\end{enumerate}

\vspace{1cm}
\underline{\textbf{Example:}}
Solve the problem:
$$\frac{du}{dx}+u=x, \ \ u(0) = 4$$

\textbf{Solution:}

\textbf{Step 1:}
The equation is already in standard form, so this step is easy.

\vspace{0.25cm}
\textbf{Step 2:} Find the integrating factor $\mu$.

$$mu = e^{-\int P(x) \ dx} = e^{-\int 1 \ dx} = e^{-x}$$

\vspace{0.25cm}
\textbf{Step 3:} Solve for the general solution $u(x)$ using Equation \ref{eq:fol-solution}
\marginnote[1.25cm]{$\leftarrow$ For the integral $\int e^x x \ dx$ we need to use integration by parts.}
\begin{align*}
u(x) &= Ce^{-x}+e^{-x}\int e^{x}x \ dx \\
&= Ce^{-x} + e^{-x}\left[xe^{x}-e^{x} \right] \\
&= Ce^{-x} + x - 1
\end{align*}

\vspace{0.25cm}
\textbf{Step 4:} Apply initial/boundary conditions if given

\begin{align*}
u(0) &= Ce^{0} + 0 -1 \\
 &=C-1 = 4 \\
 \Rightarrow C &= 5 \\
 u(x) &= 5e^x+x-1
\end{align*}



