\chapter{Assignment \#4}
\label{ch:ass4n}

\begin{fullwidth}
\begin{enumerate}
\item Determine the LU decomposition of the matrix below by hand.
\begin{equation*}
A = 
\bracketMatrixstack{
2 & 4 & 6 \\
3 & 5 & 1 \\
6 & -2 & 2 
}
\end{equation*}

\vspace{4.0cm}

\item Carry out (by hand) the first three iterations of the solution of the following system of equations using the Gauss-Seidel iterative method.  For the first guess of the solution, take the value of all unknowns to be zero.
\begin{align*}
8x_1 + 2x_2 + 3x_3 &= 51 \\
2 x_1 + 5x_2 + x_3 &= 23 \\
-3x_1 + x_2 + 6x_3 &= 20
\end{align*} 


\vspace{4.0cm}

\item Consider the linear system given below:
\begin{equation*}
\bracketMatrixstack{
4 & 0 & 1 & 0 & 1 \\
2 & 5 & -1 & 1 & 0 \\
1 & 0 & 3 & -1 & 0 \\
0 & 1 & 0 & 4 & -2 \\
1 & 0 & -1 & 0 & 5
}
\bracketVectorstack{
x_1 \\
x_2 \\
x_3 \\
x_4 \\
x_5
}
=
\bracketVectorstack{
32 \\
19 \\
14 \\
-2 \\
41
}
\end{equation*}

\begin{enumerate}
\item Find the solution using the built-in MATLAB function \lstinline[style=myMatlab]{[L,U,P] = lu(A)}---be sure to get and use the permutation matrix \lstinline[style=myMatlab]{P}.
\item Find the relative residual for your solution.
\item Find the inverse of the matrix, $A^{-1}$, using the built-in MATLAB function \lstinline[style=myMatlab]{inv(A)}, and compute the 2-norm of $A$ and $A^{-1}$ using the built-in MATLAB function \lstinline[style=myMatlab]{norm(A,p)}, where \lstinline[style=myMatlab]{p} should be set to 2 for the 2-norm.

\item Calculate the size of the error bound:
\begin{equation*}
\frac{1}{\kappa(A)}\frac{||r||}{||b||} \le \frac{||e||}{||x^{\star}||} \le \kappa(A) \frac{||r||}{||b||}
\end{equation*}
where $\kappa(A)$ is the condition number of $A$, and $r$ is the residual.

\item Repeat steps a) through d) for the test matrix BCSSTK26 from the Matrix Market.  Once you have read the matrix into MATLAB, convert the matrix to a ``full'' (non-sparse) format using the MATLAB built-in function \lstinline[style=myMatlab]{A_full = full(A_sparse)}.  or the right-hand-side vector $b$, use a vector of all ones.  Note how different the error bound is in this case.

\end{enumerate}

\vspace{8.0cm}

\item Using Jacobi, Gauss-Seidel, and SOR $(\omega = 1.4)$ iterative methods, write and run code to solve the following linear system of equations:
\begin{equation*}
\bracketMatrixstack{
7 & 3 & -1 & 2 \\
3 & 8 & 1 & -4 \\
-1 & 1 & 4 & -1 \\
2 & -4 & -1 & 6
}
\bracketVectorstack{
x_1 \\
x_2 \\
x_3 \\
x_4
}
=
\bracketVectorstack{
-1 \\
0 \\
-3 \\
1
}
\end{equation*}
The stopping criterion should be the relative change of the estimated solution in the 2-norm:
\begin{equation*}
\text{tolerance} = \frac{||x^{(k+1)} - x^{(k)}||_2}{||x^{(k)}||_2}
\end{equation*}
with tolerance set to $10^{-9}$.  Compare the number of iterations required in each case.


\vspace{3.0cm}

\item For this problem we will compare two iterative methods to solve BCSSTK26---a test matrix derived from a seismic simulation of a nuclear power plant structure.  For the right-hand-side vector $b$, use a vector of all ones.

\begin{enumerate}
\item Use the SOR method with an estimated relative error tolerance of $10^{-3}$.  Vary the relaxation parameter over the range $\omega \in [1.25,2.0)$.  Make a plot of the number of iterations required as a function of $\omega$ and approximate the best value of $\omega$ for this system.

\item Use the MATLAB built-in function \lstinline[style=myMatlab]{gmres} without a preconditioner to solve this problem.  

\item Use \lstinline[style=myMatlab]{gmres} again but with an incomplete LU preconditioner.  Use \lstinline[style=myMatlab]{opts_ilu.type='ilutp'} and \lstinline[style=myMatlab]{opts_ilu.droptol=dtol} and vary the value of \lstinline[style=myMatlab]{dtol} over the range $\text{dtol} \in [10^{-7},10^{-3}]$.  What happens as \lstinline[style=myMatlab]{dtol} gets larger? \emph{\textbf{Hint:} Use the built-in function \lstinline[style=myMatlab]{spy(L)} to show the sparsity pattern of \lstinline[style=myMatlab]{L} and note the number of non-zeros of \lstinline[style=myMatlab]{L}.  How does \lstinline[style=myMatlab]{nna(L)} change as \lstinline[style=myMatlab]{dtol} is changed?}

\item This is a relatively small sparse system and can be solved using direct methods.  Use the MATLAB built-in function \lstinline[style=myMatlab]{timeit} to estimate the time required to solve the system of equations using the built-in \lstinline[style=myMatlab]{mldivide()}---``backslash''---function.
\end{enumerate}
Organize your findings from part a), b) and c) into a short written document.  A couple of paragraphs should be sufficient.  Include the plot you created from part a) and any other graphics/tables that you find helpful to communicate what you observed in parts b) through d).

\end{enumerate}

\end{fullwidth}
