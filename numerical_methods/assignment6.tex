\chapter{Assignment \#6}
\label{ch:ass6n}

\begin{fullwidth}
\begin{enumerate}
\item Measurements of thermal conductivity, $k$ (W/m-K), of silicon at various temperatures are given in the table below.

\begin{table}[h!]
\centering
\begin{tabular}{|c|c|c|c|c|c|c|c|c|}
\hline
$\text{T} [K]$ & 50 & 100 & 150 & 200 & 400 & 600 & 800 & 1000 \\ \hline
$k$ [W/m-K] & 28 & 9.1 & 4.0 & 2.7 & 1.1 & 0.6 & 0.4 & 0.3 \\ \hline
\end{tabular}
\end{table}

The data is to be fitted with a function of the form $k=f(T)$.  Determine which of the nonlinear equations presented in Lecture 14 can best fit the data and determine its coefficients.  Make a plot that shows the data points (asterisk marker) and the equation (solid line).

\vspace{3.0cm}

\item The following data are given:

\begin{table}[h!]
\centering
\begin{tabular}{|c|c|c|c|c|c|c|}
\hline
$x$ & 1 & 2.2 & 3.4 & 4.8 & 6 & 7 \\ \hline
$y$ & 2 & 2.8 & 3 & 3.2 & 4 & 5 \\ \hline
\end{tabular}
\end{table}

Write the polynomial in Lagrange form that passes through the points.

\pagebreak

\item Values of enthalpy per unit mass, $h$, of an equilibrium Argon plasma (Ar, $\text{Ar}^{+}$, $\text{Ar}^{++}$, $\text{Ar}^{+++}$ ions and electrons) versus temperature are given in the table below.

\begin{table}[h!]
\centering
\begin{tabular}{|c|c|c|c|c|c|c|c|c|c|c|c|}
\hline
$T \times 10^3$ [K] & 5 & 7.5 & 10 & 12.5 & 15 & 17.5 & 20 & 22.5 & 25 & 27.5 & 30 \\ \hline
$h$ [MJ/kg] & 3.3 & 7.5 & 41.8 & 51.8 & 61 & 101.1 & 132.9 & 145.5 & 171.4 & 225.8 & 260.9 \\ \hline
\end{tabular}
\end{table}

Write a script that uses interpolation to calculate $h$ at temperatures ranging from 5000 K to 30000 K in increments of 500 K.  The program should generate a plot that shows the interpolated points and the data points from the table (use an asterisk marker).  
\begin{enumerate}
\item For interpolation use Lagrange polynomials as demonstrated in Lecture 15.
\item For interpolation use MATLAB's built-in tool \lstinline[style=myMatlab]{interp1} with \lstinline[style=myMatlab]{method='spline'}.
\end{enumerate}

\vspace{2.0cm}

\item Given the following data:

\begin{table}[h!]
\centering
\begin{tabular}{|c|c|c|c|c|c|}
\hline
$x$ & 1.1 & 1.2 & 1.3 & 1.4 & 1.5 \\ \hline
$f(x)$ & 0.6133 & 0.7822 & 0.9716 & 1.1814 & 1.4117 \\ \hline
\end{tabular}
\end{table}

Find the first derivative, $f^{\prime}(x)$, at the point $x=1.3$.
\begin{enumerate}
\item Use the three-point forward-difference formula.
\item Use the three-point backward-difference formula.
\item Use the two-point centered difference formula.
\end{enumerate}

\vspace{2.0cm}

\item Given the following data:

\begin{table}[h!]
\centering
\begin{tabular}{|c|c|c|c|c|c|}
\hline
$x$ & 0.6 & 0.7 & 0.8 & 0.9 & 1.0 \\ \hline
$f(x)$ & 5.2296 & 3.6155 & 2.7531 & 2.2717 & 2 \\ \hline
\end{tabular}
\end{table}

Find the second derivative, $f^{\prime \prime}(x)$, at the point $x=0.8$.
\begin{enumerate}
\item Use the three-point forward-difference formula.
\item Use the three-point backward-difference formula.
\item Use the two-point centered difference formula.
\end{enumerate}

\end{enumerate}
\end{fullwidth}
