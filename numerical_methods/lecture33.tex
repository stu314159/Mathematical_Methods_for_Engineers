\chapter{Lecture 33 - The Finite Element Method, Galerkin Method and Weak Form}
\label{ch:lec33n}
\section{Objectives}
The objectives of this lecture are to:
\begin{itemize}
\item Describe the Method of Weighted Residuals to approximately solve a simple ODE.
\item Motivate and dervie a Weak Formulation of the example problem.
\end{itemize}
\setcounter{lstannotation}{0}

\section{Example BVP}
Consider the following boundary value problem:
\begin{equation*}
\frac{d^2u}{dx^2}-u = -x, \ \ \ 0 < x < 1 
\end{equation*}
with homogeneous Dirichlet boundary conditions: $u(0)=u(1) = 0$.

\newthought{This is a} second-order, linear, non-homogeneous, equation with constant coefficients.  The complementary solution is: $u_c(x) = c_1\cosh{(x)}+c_2\sinh{(x)}$.  Using the method of undetermined coefficients, one can show that a particular solution is: $u_p = x$.  Combining the two gives us the general solution: $u(x) = u_c + u_p = x + c_1\cosh{(x)}+c_2\sinh{(x)}$. Applying boundary conditions gives us the solution:

\begin{equation}
u(x) = x - \frac{\sinh{(x)}}{\sinh{(1)}}
\label{eq:lec33n-ex-analytic}
\end{equation}
