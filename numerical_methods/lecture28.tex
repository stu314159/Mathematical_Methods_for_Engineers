\chapter{Lecture 28 - Solving Boundary value Problems Using the Shooting Method}
\label{ch:lec28n}
\section{Objectives}
The objectives of this lecture are to:
\begin{itemize}
\item Review basic concepts for Boundary Value Problems.
\item Describe the Shooting Method.
\item Do an example problem.
\end{itemize}
\setcounter{lstannotation}{0}

\section{Boundary Value Problems}
A boundary value problem (BVP) consists of a governing equation and boundary conditions.  For a second-order boundary value problem in one spatial dimension, the governing equation is as given in Equation \ref{eq:lec28n-bvp-gov-eq}:
\begin{equation}
\frac{d^2 u}{dx^2} = f\left(x,u,\frac{du}{dx}\right), \ \ a\le x \le b
\label{eq:lec28n-bvp-gov-eq}
\end{equation}
where it is understood that $a<b$.  

In order to obtain a unique solution, suitable boundary conditions must be provided.  Linear boundary conditions for the second-order problem take the form shown below:
\begin{align*}
A_1u(a)+B_1u^{\prime}(a)&=C_1 \\
A_2u(b)+B_2u^{\prime}(b)&=C_2
\end{align*}
where $A_i$, $B_i$, and $C_i$ are not \emph{all} zero for any value of $i$.
