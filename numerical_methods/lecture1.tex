\chapter{Lecture 1 - Course Introduction and Numeric Preliminaries}
\label{ch:lec1n}
\section{Objectives}
The objectives of this lecture are:
\begin{itemize}
\item Introduce the course objectives.
\item Describe how numbers are represented on computers.
\item Outline sources of errors in numerical methods relative to analytical methods.
\end{itemize}

\section{Course Introduction}

This course is intended to be an introduction and overview of numerical methods.\marginnote[-0.5cm]{\textbf{Question:} What are numerical methods? 

\vspace{0.25cm}

\textbf{Answer:} Numerical methods (or numerical analysis) is the study of \emph{algorithms} for the problems of \emph{continuous} mathematics.}  The target audience comprises undergraduate students of engineering who have already taken a full sequence of calculus and differential equations courses.  Some students may also have taken the analytical methods course described earlier in this book.  

\subsection{Objectives}
\newthought{The objectives} for this course are as follows:

\begin{enumerate}
\item Students will understand the fundamentals of numerical methods with an emphasis on the most essential algorithms and methods.

\item Students will have enhanced their scientific computing skills and have developed proficiency in using the MATLAB environment to implement algorithms and have learned to critically evaluate their results.

\item Students will understand the fundamentals of the finite element method (FEM) and have attained an introductory-level proficiency in using the COMSOL software package to carry out a multi-physics analysis of a relevant physical model.

\item Students will have developed the ability to formulate a problem of interest, apply numerical methods and computational tools to analyze the problem and communicate pertinent results to others.
\end{enumerate}


\subsection{Course Topics}
The specific topics we will cover include:

\begin{enumerate}
\item Methods for solving non-linear equations
\begin{enumerate}
\item Bisection method
\item Newton's method
\item Secant method

\end{enumerate}
\item Methods for solving linear equations
\begin{enumerate}
\item Gauss elimination and related methods like LU- and Cholesky factorization
\item Iterative solution methods for sparse linear systems of equations\sidenote{We will also discuss some preconditioners.}
\end{enumerate}
\item Curve fitting
\begin{enumerate}
\item Least squares algorithms
\item Curve fitting with Lagrange polynomials
\end{enumerate}
\item Numeric differentiation
\begin{enumerate}
\item Finite difference methods
\item Lagrange polynomials
\end{enumerate}
\item Numeric integration
\begin{enumerate}
\item A variety of Newton-Cotes methods
\item Gauss Quadrature
\end{enumerate}

\item Initial- and Boundary-value problems \sidenote{This section will prominently include MATLAB built-in methods; particularly for boundary value problems.}
\begin{enumerate}
\item Runge-Kutta methods for initial value problems\sidenote{We will not extensively cover either explicit or implicit multi-step methods, giving preference to the wide variety of very effective RK methods.}
\item Shooting method for boundary value problems
\item Finite element method\sidenote{There will be a simple MATLAB demonstration with application to one-dimensional boundary value problems.  The majority of FEM coverage will be related to the use of COMSOL.}
\end{enumerate}

\end{enumerate}  

\vspace{4.0cm}

\section{Representation of Numbers on a Computer}

Every engineer who uses computational tools in their work should have a basic understanding of how numbers are represented on a computer.

