\chapter{Lecture 6 - Newton's Method for Systems of Non-Linear Equations}
\label{ch:lec6n}
\section{Objectives}
The objectives of this lecture are to:
\begin{itemize}
\item Describe Newton's method for finding roots to systems of non-linear equations.
\item Do an example problem.
\item Describe MATLAB functions for solving a system of non-linear equations.
\end{itemize}
\setcounter{lstannotation}{0}

\section{Newton's Method for Systems of Non-Linear Equations}

In Lecture 4 we derived Newton's method for finding the root to a single non-linear equation.  The basic iteration procedure was to update our estimate of the root according to the equation:
\begin{equation*}
x_{i+1} = x_i - \frac{f(x_i)}{f^{\prime}(x_i)}
\end{equation*}
The idea on which this was derived was to project a line tangent to $f(x_i)$ to the $x$-axis. 

\index{Taylor series expansion}
\newthought{An alternative way} to derive this update relation is based on the Taylor series expansion.  In a Taylor series expansion, given a function evaluated at some point, $f(x_1)$, we estimate the value of the function evaluated at some \emph{other} point, $x_2$, using the equation below:

\begin{align*}
f(x_2) &= f(x_1) +  \frac{f^{\prime}(x_1)}{1!}(x_2 - x_1) +  \frac{f^{\prime \prime}(x_1)}{2!}(x_2 - x_1)^2 + \cdots \\
&= \sum\limits_{n=0}^{\infty} \frac{f^{(n)}(x_1)}{n!}(x_2 - x_1)^n
\end{align*}
If we ignore terms proportional to $f^{\prime \prime}(x_1)$ and higher, and if we suppose that $x_2$ is a root, we get:
\begin{align*}
f(x_2) = 0 &= f(x_1) + f^{\prime}(x_1)(x_2 - x_1) \\
x_2 - x_1 &= \frac{-f(x_1)}{f^{\prime}(x_1)} \\
\Rightarrow x_2 &= x_1 - \frac{f(x_1)}{f^{\prime}(x_1)} 
\end{align*}
which is the same as what we started with.

\newthought{We would now} like to generalize Newton's method to find roots for a \emph{system} of two or more equations.  We will use the formulation based on the Taylor series expansion to do this.  

Consider the case of 2 non-linear functions of 2 variables:
\begin{equation*}
f_1(x,y) = 0 , \ \ f_2(x,y) = 0
\end{equation*}
If $x_2$ and $y_2$ are the true (unknown) solution to the equations, and $x_1$ and $y_1$ are points sufficiently close to the solution then:
\begin{align*}
f_1(x_2,y_2) &= 0 = f_1(x_1,y_1) + (x_2 - x_1) \frac{\partial f_1}{\partial x}\Bigr|_{x_1,y_1} + (y_2 - y_1) \frac{\partial f_1}{\partial y}\Bigr|_{x_1,y_1}  \\
f_2(x_2,y_2) &= 0 = f_2(x_1,y_1) + \underbrace{(x_2 - x_1)}_{\Delta x} \frac{\partial f_2}{\partial x}\Bigr|_{x_1,y_1} + \underbrace{(y_2 - y_1)}_{\Delta y} \frac{\partial f_2}{\partial y}\Bigr|_{x_1,y_1} \\
\end{align*}
where higher order terms are neglected.  This set of linear equations can be expressed in a matrix-vector format as shown in Equation \ref{eq:lec6n-matvec}.

\index{Jacobian}
\begin{equation}
\bracketMatrixstack{
\frac{\partial f_1}{\partial x}\Bigr|_{x_1,y_1} & \frac{\partial f_1}{\partial y}\Bigr|_{x_1,y_1} \\
\frac{\partial f_2}{\partial x}\Bigr|_{x_1,y_1} & \frac{\partial f_2}{\partial y}\Bigr|_{x_1,y_1}
}
\bracketVectorstack{
\Delta x \\
\Delta y
}
=
\bracketVectorstack{
-f_1(x_1,y_1) \\
-f_2(x_1,y_1)
}
\label{eq:lec6n-matvec}
\end{equation}
The unknown quantities are $\Delta x$ and $\Delta y$.  The matrix in Equation \ref{eq:lec6n-matvec} is referred to as the \emph{Jacobian}. We can solve this linear system of equations\sidenote{Actually we have not covered how to do that yet in this class.  Students can probably solve this 2x2 system of equations based on high school math, but we will thoroughly study methods for solving linear systems of equations in the next section of the text.} to find $\Delta x$ and $\Delta y$ and thus get $x_2$ and $y_2$ from Equation \ref{eq:lec6n-xy-update}.
\begin{equation}
x_2 = x_1 + \Delta x, \ \ \ y_2 = y_1 + \Delta y
\label{eq:lec6n-xy-update}
\end{equation}

\vspace{4.0cm}

\newthought{In summary,} Newton's method for solving a system of non-linear equations is made up of the following steps:
\begin{enumerate}
\item Start with an initial guess, $x_1$, $y_1$.
\item Form and solve the linear system of equations given in Equation \ref{eq:lec6n-matvec} to obtain $\Delta x$ and $\Delta y$.
\item compute $x_{i+1}$ and $y_{i+1}$ from: $x_{i+1} = x_{i} + \Delta x, \ \ \ y_{i+1} = y_{i} + \Delta y$.
\item Repeat steps \#2 and \#3 until $\Delta x$ and $\Delta y$ are within some specified error tolerance.\sidenote{This relative error tolerance will be illustrated in the example code.}
\end{enumerate}

\vspace{0.25cm}

\noindent\textbf{Example:} The equations of the catenary curve and the ellipse, which are shown in the figure (add figure), are given by:
\begin{align*}
f_1(x,y) &= y-\frac{1}{2}\left(e^{\sfrac{x}{2}} + e^{-\sfrac{x}{2}}\right) \\
f_2(x,y) &= 9x^2 + 25y^2 - 225
\end{align*}
Use Newton's method to determine the point of intersection of the curves that resides in the first quadrant of the coordinate system.

\vspace{0.15cm}

\noindent The Jacobian for this system of equations is given by:
\begin{equation*}
\bracketMatrixstack{
-\frac{1}{4}\left(e^{\sfrac{x}{2}} + e^{-\sfrac{x}{2}}\right) & 1 \\
18x & 50y 
}
\end{equation*}

\vspace{0.15cm}

\noindent We begin by clearing out the workspace and defining the given functions:
\begin{lstlisting}[style=myMatlab, name=lec6n-ex1]
clear
clc
close 'all'

F1 = @(x,y) y - 0.5*(exp(x./2) + exp(-x./2));
F2 = @(x,y) 9*x.^2 + 25*y.^2 - 225;

dF1x = @(x,y) -0.25*(exp(x./2) - exp(-x./2));
dF1y = @(x,y) 1;

dF2x = @(x,y) 18*x;
dF2y = @(x,y) 50*y;

Jac = @(x,y) [dF1x(x,y) dF1y(x,y); dF2x(x,y) dF2y(x,y)];
\end{lstlisting}

\vspace{0.15cm}

\noindent Next we will define $x_1$ and $y_1$ and specify our stopping criteria.
\begin{lstlisting}[style=myMatlab,name=lec6n-ex1]
xi = 2.5; yi = 2; 
Err = 1e-10; imax = 10;
\end{lstlisting}
Note that we do not plan on making many iterations.  If Newton's method works, we will very quickly converge.
\vspace{4.0cm}

\noindent Now we implement Newton's method:\marginnote[1.0cm]{
\ref{lst:ann6n-1} We use MATLAB's built-in ``backslash'' operator to solve the linear system of equations.  This will be discussed in Section VIII of this text.

\vspace{0.3cm}


\ref{lst:ann6n-2} Here we compare $\sfrac{\Delta x}{x}$ and $\sfrac{\Delta y}{y}$ for our relative error tolerance.  We are really not computing the error, we are computing how small our updates are to $x$ and $y$.

\vspace{0.5cm}

\ref{lst:ann6n-3} The operator \lstinline[style=myMatlab]{&&} is the logical \emph{and} operator.

}
\begin{lstlisting}[style=myMatlab,name=lec6n-ex1]
for i = 1:imax
   J = Jac(xi,yi);
   F = -[F1(xi,yi); F2(xi,yi)];
   dp = J\F;   /*!\annotation{lst:ann6n-1}!*/
   xip = xi + dp(1);
   yip = yi + dp(2);
   Err_x = abs((xip - xi)/xi);  /*!\annotation{lst:ann6n-2}!*/
   Err_y = abs((yip - yi)/yi);
   
   fprintf('i = %i  x = %-7.4f  y = %-7.4f  Error in x = %-7.4g Error in y = %-7.4g \n',...
       i,xip,yip,Err_x,Err_y);
   
   if (Err_x < Err) && (Err_y < Err)  /*!\annotation{lst:ann6n-3}!*/
       break
   else
       xi = xip; yi = yip;
   end
    
end
\end{lstlisting}
This script converges to the solution $x = 3.0311553917$ and $y=2.38586565356$ in 5 iterations.

\section{Implementation with FSOLVE}
The primary MATLAB built-in function you should use for solving a system of non-linear equations is \lstinline[style=myMatlab]{fsolve}.  The basic syntax for using \lstinline[style=myMatlab]{fsolve} is:
\begin{center}
\begin{tabular}{c}
\begin{lstlisting}[style=myMatlab, frame=none, numbers=none, basicstyle=\small]
x = fsolve(fun,x0);
\end{lstlisting}
\end{tabular}
\end{center}
A more complete syntax that provides additional output information and an interface for passing options to the algorithm is:
\begin{center}
\begin{tabular}{c}
\begin{lstlisting}[style=myMatlab, frame=none, numbers=none, basicstyle=\small]
[x, fval, exitflag, output, jacobian] = ...
     fsolve(fun,x0,options);
\end{lstlisting}
\end{tabular}
\end{center}
The values of \lstinline[style=myMatlab]{x} and \lstinline[style=myMatlab]{fval} are similar to what one obtains when using \lstinline[style=myMatlab]{fzero} except, of course they are now both vectors.  Values for \lstinline[style=myMatlab]{exitflag} and \lstinline[style=myMatlab]{output} are different---interested readers should consult the MATLAB documentation---but \lstinline[style=myMatlab]{exitflag=1} still means success.  As with \lstinline[style=myMatlab]{fzero} a function is used to construct an appropriate \lstinline[style=myMatlab]{options} structure.  When using \lstinline[style=myMatlab]{fsolve} you should use the \lstinline[style=myMatlab]{optimoptions} function. An example usage of \lstinline[style=myMatlab]{optimoptions} is included in the MATLAB listing below.

\vspace{3.0cm}

\begin{lstlisting}[name=lec6n-ex2, style=myMatlab]
clear
clc
close 'all'

F = @(x) ex3p5(x);
X0 = [2.5, 2.0];

option_set = 2;
% 1 = no outputs
% 2 = detailed outputs

switch option_set
    
    case 1
        options = optimoptions('fsolve','Display','none');
        
    case 2
        options = optimoptions('fsolve','Display',...
            'iter-detailed','MaxIterations',1000,...
            'StepTolerance',1e-10);
        
    otherwise % default options
        options = optimoptions('fsolve');
        
end

[x,fval,exitflag,output] = fsolve(F,X0,options);

fprintf('Root found at x = %8.7g, y = %8.7g \n',...
    x(1),x(2));
fprintf('fval = \n'); disp(fval);
fprintf('exitflag = %d \n',exitflag);

%% Local Function
function out = ex3p5(x)
[m,n] = size(x); % expect scalar or vector input
assert(min(m,n)==1,...
    'Error!  Vector input expected for x! \n');
out = nan(m,n); % construct output

out(1) = x(2) - 0.5*(exp(x(1)./2) + exp(-x(1)./2));
out(2) = 9*x(1).^2 + 25*x(2).^2 - 225;

end
\end{lstlisting}
  

