\chapter{Lecture 35 - Galerkin FEM in 1D with Conduction and Convection Boundary conditions}
\label{ch:lec35n}
\section{Objectives}
The objectives of this lecture are to:
\begin{itemize}
\item Illustrate the application of the FEM for problems with type-2 and type-3 boundary conditions.
\item Discuss issues relevant for more analysis of more general partial differential equations with the FEM.
\end{itemize}
\setcounter{lstannotation}{0}

\section{Problem Statement}
In the last lecture we solved a problem that had only type-1 boundary conditions.  In this lecture we will show how to solve problems with type-2 and type-3 boundary conditions.  

For this exercise we will re-solve a problem that we first met in Lecture 30. Consider the nuclear fuel pin depicted in Figure \ref{fig:lec35n-ex1-schematic}.  As described in Lecture 30,  heat is generated by fission in the cylindrical fuel pellet; this heat is conducted through the fuel and cladding and dissipated to coolant flowing along the outside of the fuel pin.  Prompt fission gamma rays also deposit energy in the cladding due to photon interactions with the clad material.  The outer surface of the cladding is cooled by flowing water at $T_{\infty}=473$K with heat transfer coefficient of $h=10^4\ \text{W/m}^2\text{-K}$.  The thermal conductivity of the cladding material is $k=16.75 \ \text{W/m-K}$.  The dimensions of the fuel rod are $R=1.5\times 10^{-2}\text{ m}$, and $w=3.0\times 10^{-3}\text{ m}$.  At steady-state, conservation of energy in the cladding can be expressed by the following boundary value problem:
\begin{marginfigure}
\includegraphics{lec30n-ex1-schematic.png}
\caption{A typical nuclear reactor fuel pin.}
\label{fig:lec35n-ex1-schematic}
\end{marginfigure}
\begin{table}[h!]
\begin{tabular}{l l}
ODE: & $\frac{1}{r}\frac{d}{dr}\left(rk\frac{dT}{dr}\right)=-10^8\frac{e^{-r/R}}{r}, \ \  R < r < R+w $ \\
BCs: & $\frac{dT}{dr}\Bigl|_{r=R}=-\frac{6.32\times 10^5}{k}, \ \ \frac{dT}{dr}\Bigl|_{r=R+w}=-\frac{h}{k}\left(T(r+w)-T_{\infty}\right)$ \\
\end{tabular}
\end{table}

\newthought{Recall that in} Lecture 30, we re-expressed the governing equation so it would be ammenable to solution with \lstinline[style=myMatlab]{bvp5c}.  As a result, the governing equation can be equivalently written as shown in Equation \ref{eq:lec35n-gov-eq}.
\begin{equation}
\frac{d^2T}{dr^2}+\frac{1}{r}\frac{dT}{dr} = -\frac{10^8e^{-r/R}}{rk}
\label{eq:lec35n-gov-eq}
\end{equation}

We start by expressing this equation in the strong form of the weighted residual statement:\marginnote{

\noindent\textbf{Note:} We change our convention and denote the test function with $v$ since the character $w$ is already used to indicate the clad thickness.

}
\begin{equation*}
\int_{R}^{R+w} v \frac{d^2T}{dr^2} \ dr + \int_{R}^{R+w} v \frac{1}{r}\frac{dT}{dr} \ dr + \int_{R}^{R+w} v \frac{10^8e^{-r/R}}{rk} \ dr = 0
\end{equation*}
We convert this to the weak form by carrying out integration by parts on the first term.  The result is:
\begin{multline*}
\underbrace{v(R+w) \frac{dT}{dr}\Bigl|_{r=R+w} - v(R)\frac{dT}{dr}\Bigl|_{r=R}}_{\text{boundary terms}} \\
-\int_{R}^{R+w} \frac{dv}{dr}\frac{dT}{dr} \ dr +  \int_{R}^{R+w} v \frac{1}{r}\frac{dT}{dr} \ dr + \int_{R}^{R+w} v \frac{10^8e^{-r/R}}{rk} \ dr = 0
\end{multline*}

\noindent We can substitute the given boundary conditions into the boundary terms of the weak form:
\begin{multline*}
\underbrace{v(R+w) \left[-\frac{6.32\times 10^5}{k} \right] - v(R)\left[-\frac{h}{k}\left(T(r+w)-T_{\infty}\right) \right]}_{\text{boundary terms}} \\
-\underbrace{\int_{R}^{R+w} \frac{dv}{dr}\frac{dT}{dr} \ dr}_{\text{kernel \#1}} +\underbrace{\int_{R}^{R+w} v \frac{1}{r}\frac{dT}{dr} \ dr}_{\text{kernel \#2}} +\underbrace{\int_{R}^{R+w} v \frac{10^8e^{-r/R}}{rk} \ dr}_{\text{source term}} = 0
\end{multline*}
Now the problem is in a form that can be solved using the FEM.

\section{Matlab Solution}
Almost all of the code used for the second example in Lecture 34 can be re-used.  We start by clearing out the workspace and introducing problem-specific parameters.  Almost all of the remainder of the code is re-used.  The only part that needs to be specialized is for calculation of, what we have dubbed, the kernels and the source term.  



\begin{lstlisting}[style=myMatlab,name=lec35n-ex]
clear
clc
close 'all'

%% Parameters
R = 1.5e-2; % m, radius of fuel
w = 3.0e-3; % m, thickness of cladding
k = 16.75; % W/(m-K), thermal conductivity of clad
Q = 1e8; % W/m^2, Source term from heat dep in cladding.
Q2 = 6.32e5; % W/m^2, Heat flux due to heat produced in fuel.
T_inf = 423; % K, temperature of water flowing on cladding
h = 1e4; % W/(m^2-K), convective heat transfer coefficient

a = R; b = R+w;
\end{lstlisting}

\marginnote{

\vspace{3.5cm}

\noindent\ref{lst:ann35n-1} This portion of the listing is completely re-used.

}
\begin{lstlisting}[style=myMatlab,name=lec35n-ex]
% initialize global arrays
K1 = zeros(nnodes,nnodes);
K2 = zeros(nnodes,nnodes);
S1 = zeros(nnodes,1);

% carry out assembly process
for ele = 1:nelem

    % local arrays to be populated  /*!\annotation{lst:ann35n-1}!*/
    k1 = zeros(nldofs,nldofs);
    k2 = zeros(nldofs,nldofs);
    s1 = zeros(nldofs,1);
    
    % local mapping for GQ
    aL = gcoord(nodes(ele,1));
    bL = gcoord(nodes(ele,end));
    xT = @(t) ((bL - aL)*t + aL + bL)/2;
    Jac = (bL - aL)/2;

    % Get sample points for shape functions
    xgl = gcoord(nodes(ele,:));
    
    % get Lagrange Interpolant of requested order
    H = getLagrangeInterp(xgl);
    Hp = getLagrangeInterpDeriv(xgl);
\end{lstlisting}

In this next section, we carry out the calculations for the elements of the weak form of the minimum weighted residual statement.

\marginnote{

\vspace{1.4cm}

\noindent\ref{lst:ann35n-2} We introduce this variable: a) to improve readability; and b) to avoid re-computing \lstinline[style=myMatlab]{xT(q(qp))} numerous times within the loop.

\vspace{0.1cm}

\noindent\ref{lst:ann35n-3} Compute the contribution to kernel \#1.

\vspace{0.3cm}

\noindent\ref{lst:ann35n-4} Compute the contribution to kernel \#2.

\vspace{0.6cm}

\noindent\ref{lst:ann35n-5} Compute the contribution to the source term.
}
\begin{lstlisting}[style=myMatlab,name=lec35n-ex]
    for qp = 1:nqp
        
        % sum weighted contribution at Gauss Points
        xqp = xT(q(qp)); /*!\annotation{lst:ann35n-2}!*/
        for i = 1:nldofs
            for j = 1:nldofs
                
                k1(i,j) = k1(i,j) + ...    /*!\annotation{lst:ann35n-3}!*/
                    Hp{i}(xqp)*Hp{j}(xqp)*w(qp);
                                
                k2(i,j) = k2(i,j) + ...    /*!\annotation{lst:ann35n-4}!*/
                    (1./xqp)*H{i}(xqp)*Hp{j}(xqp)*w(qp);
            end % j
            c = (1./xqp)*(1/k)*Q*exp(-xqp/R);
            s1(i) = s1(i) + c*H{i}(xqp)*w(qp);  /*!\annotation{lst:ann35n-5}!*/
        end % i
     
    end % qp
    % apply Jacobian to map to physical coordinates
    k1 = k1*Jac;
    k2 = k2*Jac;
    s1 = s1*Jac;

\end{lstlisting}
Now that the subarrays for the kernels and source term are computed for the element, we add them to the global arrays.

\begin{lstlisting}[style=myMatlab,name=lec35n-ex]
    % add local arrays to global arrays "assembly"
    for i = 1:nldofs
        for j = 1:nldofs
            row = nodes(ele,i); col = nodes(ele,j);
            K1(row,col) = K1(row,col) + k1(i,j);
            K2(row,col) = K2(row,col) + k2(i,j);
        end % j
        dof = nodes(ele,i);
        S1(dof) = S1(dof) + s1(i);
    end % i

end % numel

% apply boundary conditions
A = -K1 + K2;
RHS = -S1;

% provide contributions from boundary conditions
c1 = Q2/k;
RHS(1)=RHS(1)-c1;

c2 = h/k;
RHS(nnodes)=RHS(nnodes)-c2*(T_inf);
A(nnodes,nnodes) = A(nnodes,nnodes)-c2;

% solve the system of equations
u = A\RHS;
\end{lstlisting}



