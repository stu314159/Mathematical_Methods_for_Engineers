\chapter{Lecture 1 - Introduction, Definitions and Terminology}
\label{ch:lec1}%<< think of better label and chapter names
\section{Objectives}
The objectives of this lecture are:
\begin{itemize}
\item Provide an overview of course content
\item Define basic terms related to differential equations
\item Provide examples of classification schemes for differential equations
\end{itemize}

\section{Course Introduction}
\newthought{This course is intended} as a one-semester introduction to partial differential equations.  It is assumed that all students have a thorough background in single- and multi-variable calculus as well as differential equations.  The first few lectures comprises a review of the portions of differential equations on which this course most heavily relies.  This is followed by a treatment of power series methods and the method of Frobeneius.  These are needed so that students will understand the origins of Legendre Polynomials and Bessel functions that will be used in the solution of boundary value problems in spherical and cylindrical coordinates respectively.

\newthought{The main body} of material deals with the solution of (mostly homogeneous) boundary value problems---wave equation, heat equation, and laplace equation---in rectangular, polar/cylindrical, and spherical coordinate systems.  For this a preparatory review of Fourier series expansions along with Fourier-Legendre and Fourier-Bessel expansions are introduced along with a levening of Sturm-Liouville theory in boundary value problems.  The rest is a problem-by-problem tour of methods and analysis with heavy emphasis on heat transfer and nuclear engineering applications.

\section{Classification of Differential Equations}

\subsection{Classification by Type}

\subsection{Classification by Order}

\subsection{Classification by Linearity}

\section{Verification of a Solution} 


