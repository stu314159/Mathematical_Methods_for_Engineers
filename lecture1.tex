\chapter{Lecture 1 - Introduction, Definitions and Terminology}
\label{ch:lec1}%<< think of better label and chapter names
\section{Objectives}
The objectives of this lecture are:
\begin{itemize}
\item Provide an overview of course content
\item Define basic terms related to differential equations
\item Provide examples of classification schemes for differential equations
\end{itemize}

\section{Course Introduction}
\newthought{This course is intended} as a one-semester introduction to partial differential equations.  It is assumed that all students have a thorough background in single- and multi-variable calculus as well as differential equations.  The first few lectures comprise a review of the portions of differential equations on which this course most heavily relies.  This is followed by a treatment of power series methods and the method of Frobenius.  These are needed so that students will understand the origins of Legendre Polynomials and Bessel functions that will be used in the solution of boundary value problems in spherical and cylindrical coordinates respectively.

\newthought{The main body} of material deals with the solution of (mostly homogeneous) boundary value problems---wave equation, heat equation, and Laplace equation---in rectangular, polar/cylindrical, and spherical coordinate systems.  For this a preparatory review of Fourier series expansions along with Fourier-Legendre and Fourier-Bessel expansions are introduced along with a leavening of Sturm-Liouville theory in boundary value problems.  The rest is a problem-by-problem tour of methods and analysis with heavy emphasis on heat transfer and nuclear engineering applications.

\section{Classification of Differential Equations}
\newthought{It is important} to be able to classify differential equations.  In this class we will learn a variety of techniques to find the function that satisfies a differential equation along with its boundary or initial conditions.\sidenote{Consider the differential equation: $\frac{du}{dx} = ux$.  The variable $u$ stands for the function, $u(x)$, that satisfies the equation; $u$ is also referred to as the \textbf{dependent variable}.  The variable $x$ is the \textbf{independent variable}.  By convention we will use the variables $x,y,z$ and $r,\theta,\phi$ as spatial independent variables and $t$ as an independent variable for time dependent problems.  We will use many other letters to denote dependent variables but most commonly $u,v,$ and $w$.} The techniques we learn in this class are tailored for specific classes of problems; you classify the problem and that tells you what method to use.  If you improperly classify the equation, you will likely use an inappropriate method and may have trouble figuring out why it is not working.  
\subsection{Classification by Type and Order}
\newthought{We shall start} with the easiest classification categories: type and order.  There are two \emph{types} of differential equations that we will consider: ordinary differential equations; and partial differential equations.  
\newthought{In an ordinary} differential equation\marginnote{\textbf{Example ODE:} $$\frac{d^2u}{dt^2}+t \frac{du}{dt}=3e^{-t}$$  There is one independent variable, $t$}, there is only one independent variable.  In a \emph{partial} differential equation,\marginnote{\textbf{Example PDE:} $$\frac{\partial^2u}{\partial x^2} + \frac{\partial^2u}{\partial y^2} = 0 $$ There are two independent variables, $x,$ and $y$.} there are multiple independent variables and consequently derivatives of the dependent variable will partial derivatives.
\newthought{The order of} a differential equation is the order of the highest derivative in the equation.  This is typically not confusing for students. If anything needs to be added here it is to be mindful of the difference between a higher order derivative and an exponent.  For example, in the second order, non-linear, ordinary differential equation shown below, 
$$ \frac{d^2u}{dx^2}+5\left(\frac{du}{dx} \right)^3 - 4u = e^{x}  $$
it isn't \emph{too} hard to realize that the ``3'' is an exponent and the ``2'' denotes a second derivative. Still, be mindful.   

\subsection{Classification by Linearity}
An $n$-th order ordinary differential equation is said to be \emph{linear} when it can be written in the form shown in Equation \ref{eq:lin_ode}:
\begin{equation}
a_n(x)u^{(n)}+a_{n-1}(x)u^{(n-1)}+\cdots+a_1(x)^{\prime}+a_0(x)u = g(x)
\label{eq:lin_ode}
\end{equation}
The key features that you should note in the form of Equation \ref{eq:lin_ode} are:
\begin{enumerate}
\item The \emph{dependent} variable and \emph{all of its derivatives} are of the first degree; that is, the power of each term involving $u$ is 1.
\item The coefficients of each term, $a_n(x)$, depend at most on the \emph{independent} variable.
\end{enumerate}
A lot of students struggle with discriminating between linear and nonlinear ODEs but it really is as simple as checking these two things.\marginnote{\textbf{Why is this important?} Most of the techniques we will learn in this course \emph{depend} upon the fact that the equation we are trying to solve is \emph{linear}.  In \emph{``the wild''} you may be presented with (or, more likely \emph{derive}) an equation and may not be explicitly told whether or not the equation is linear.  If the equation is \textbf{not} linear you will find that most of the tools you learn in this course will not be applicable; you will most likely need to use a numerical method. You need to be able to tell the difference so you know what tools to use.}  If both conditions are satisfied; the equation is linear.  If not, the equation is nonlinear.  As examples, Equation \ref{eq:nonlin_1} violates the first criterion; Equation \ref{eq:nonlin_2} violates the second.
\begin{equation}
\frac{d^2u}{dx^2}+u^2 = 0 
\label{eq:nonlin_1}
\end{equation}
\begin{equation}
\frac{d^3u}{dx^3}-5u\frac{du}{dx}=x
\label{eq:nonlin_2}
\end{equation}
\section{Verification of an Explicit Solution} 
A solution in which the dependent variable is expressed solely in terms of the independent variable and constants is said to be an \emph{explicit} solution.\marginnote{\textbf{Example explicit solution:} $u(x) = f(x)$.} Otherwise, the solution is \emph{implicit}.\marginnote{\textbf{Example implicit solution:} $G(x,u) = 0$} 

\newthought{In this class} we will mainly be interested in finding explicit solutions to differential equations that we are given or have derived.  There are some cases, however, where we are given a function and we wish to verify that it is a solution to a given differential equation.  To do this, we simply ``plug'' the equation into the differential operator and verify that an identity is derived.

\begin{example}
\textbf{Example:} Verify that $u = \frac{6}{5} - \frac{6}{5}e^{-20t}$ is a solution to:
$$\frac{du}{dt} + 20u = 24$$
\vspace{0.5cm}
\textbf{Solution:}
Since $\frac{du}{dt} = \frac{d}{dt}(\frac{6}{5} - \frac{6}{5}e^{-20t}) = 24e^{-20t}$, we can see that:
\begin{align*}
\frac{du}{dt} + 20u &= 24e^{-20t} + 20(\frac{6}{5} - \frac{6}{5}e^{-20t}) \\
&= 24e^{-20t} + 24 - 24e^{-20t} \\
&=24
\end{align*} 
which is the expected identity.
\end{example} 

