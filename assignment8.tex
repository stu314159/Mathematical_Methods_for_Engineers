\chapter{Assignment \#8}
\label{ch:ass8}
\begin{fullwidth}
Use separation of variables to find, if possible, product solutions for the given partial differential equations.  Be sure to consider cases for all possible values of the separation constant.

\begin{enumerate}
\item $\frac{\partial u}{\partial x} = \frac{\partial u}{\partial y}$

\vspace{1.0cm}

\item $\alpha^2 \frac{\partial^2 u}{\partial x^2} - u = \frac{\partial u}{\partial t}, \ \alpha>0$

Note: for this problem, when separating variables, divide by $\alpha^2 X(x)T(t)$ and keep all terms with $\alpha^2$ together.
\end{enumerate}

\vspace{1.0cm}

\noindent Classify the given partial differential equation as hyperbolic, parabolic, or elliptic.
\begin{enumerate}[resume]
\item $\frac{\partial^2 u}{\partial x^2} + \frac{\partial^2 u}{\partial x \partial y} + \frac{\partial^2 u}{\partial y^2} = 0$

\vspace{1.0cm}

\item $\frac{\partial^2 u}{\partial x^2}=0\frac{\partial^2 u}{\partial x \partial y}$

\vspace{1.0cm}

\item $\frac{\partial^2 u}{\partial x^2} + 2 \frac{\partial^2 u}{\partial x \partial y} + \frac{\partial^2 u}{\partial y^2} + \frac{\partial u}{\partial x} - 6 \frac{\partial u}{\partial y} = 0$
\end{enumerate}

\vspace{1.0cm}

\noindent Show that the given partial differential equation possesses the indicated product solution.
\begin{enumerate}[resume]
\item $\frac{\partial u}{\partial t} = k\left(\frac{\partial^2 u}{\partial r^2} + \frac{1}{r} \frac{\partial u}{\partial r} \right)$, $ \ \ u(r,t) = e^{-k \alpha^2 t}\left(c_1 J_0(\alpha r) + c_2 Y_0(\alpha r) \right)$ 

\end{enumerate}

\vspace{1.0cm}

\noindent For the following problems, a rod of length $L$ coincides with the interval $[0,L]$ on the x-axis.  Set up the boundary-value problem for the temperature $u(x,t)$.

\begin{enumerate}[resume]
\item The left end is held at temperature zero and the right end is insulated.  The initial temperature is $f(x)$ throughout.

\vspace{1.0cm}

\item The left end is at temperature $\sin{\left(\sfrac{\pi t}{L} \right)}$, the right end is held at zero, and there is heat transfer from the lateral surface of the rod into the surrounding medium held at temperature zero.  The initial temperature is $f(x)$ throughout.
\end{enumerate}

\vspace{1.0cm}

\noindent For the following problems a string of length $L$ coincides with the interval $[0,L]$ on the x-axis.  Set up the boundary-value problem for the displacement $u(x,t)$.

\begin{enumerate}[resume]
\item The ends are secured to the x-axis. The string is released from rest from the initial displacement $u(x,0) = x(L-x)$.  

\vspace{1.0cm}

\item The left end is secured to the x-axis but the right end moves in a transverse manner according to $\sin{(\pi t)}$.  The string is released from rest from the initial displacement $f(x)$.  For $t>0$ the transverse vibrations are damped with a force proportional to the transverse velocity of the string.
\end{enumerate}


\vspace{1.0cm}

\noindent For the next problem, set up the boundary-value problem for a steady-state temperature $u(x,y)$. 

\begin{enumerate}[resume]
\item A thin rectangular plate coincides with the region in the xy-plane defined by: $0\le x \le 4, \ \ 0 \le y \le 2$.  The left end and the bottom of the plate are insulated.  The top of the plate is held at temperature zero, and the right end of the plate is held at a temperature $f(y)$. 
\end{enumerate}

\end{fullwidth}
