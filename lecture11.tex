\chapter{Lecture 11 - Solutions about Singular Points}
\label{ch:lec11}
\section{Objectives}
The objectives of this lecture are:
\begin{itemize}
\item Define regular and irregular singular points and give examples of their classification.
\item Describe the Extended Powre Series Method (Method of Frobenius)
\item Do an example problem
\end{itemize}

\section{Definitions}
Consider a linear, homogeneous, second-order differential equation in standard form as shown below:
\begin{equation*}
u^{\prime \prime} + P(x)u^{\prime} + Q(x)u = 0
\end{equation*}

\begin{definition}[Singular Point]
A \emph{singular point}, $x_0$, is a point where $P(x)$ or $Q(x)$ are not analytic.
\end{definition}

\begin{definition}[Regular/Irregular Singular Point]
A singular point $x_0$ is said to be a \emph{regular} singular point of the differential equation if the functions $p(x)=(x-x_0)P(x)$ and $q(x)=(x-x_0)^2Q(x)$ are both analytic at $x_0$.  If a singular point is not regular, it is \emph{irregular}.
\end{definition}

\noindent\textbf{Example:} Classify the singular points of $(x^2-4)^2u^{\prime\prime}+3(x-2)u^{\prime}+5u = 0$

\vspace{0.5cm}

\noindent In standard form, $P(x) = \frac{3(x-2)}{(x^2-4)^2} = \frac{(3(x-2)}{(x+2)^2(x-2)^2}$; and $Q(x) = \frac{5}{(x+2)^2(x-2)^2}$.  There are two singular points: -2 and 2.\marginnote{$x_0=2:\ \ p(x) = \frac{\cancel{(x-2)}3\cancel{(x-2)}}{(x+2)^2 \cancel{(x-2)^2}}$, so $p(x) = \frac{3}{(x+2)^2}$ which is analytic at $x=2$.

\vspace{0.25cm}

$x_0=-2: \ \ p(x) = \frac{(x+2)3(x-2)}{(x+2)^2(x-2)^2}$ so $p(x) = \frac{3(x-2)}{(x+2)(x-2)}$.}
Work is shown in the margin for $p(x)$.  From the work in the margin it should be clear for this problem that $q(x)$ is analytic at both $x=-2$ and $x=2$ but since $p(x)$ is not analytic at $x_0=-2$, $x_0=-2$ is an irregular singular point and $x_0=2$ is a regular singular point.

\begin{theorem}[Frobenius' Theorem]
If $x=x_0$ is a \emph{regular} singular point then at least one non-zero solution of the form:
$$u(x) = (x-x_0)^r\sum\limits_{n=0}^{\infty}c_n(x-x_0)^n = \sum\limits_{n=0}^{\infty}c_n(x-x_0)^{n+r}$$
where $r$ is to be determined.  The series will converge at least on some radius of convergence defined by: $0<x-x_0<R$.
\end{theorem}
