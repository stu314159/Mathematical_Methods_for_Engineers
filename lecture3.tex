\chapter{Lecture 3 - Theory of Linear Equations}
\label{ch:lec3}
\section{Objectives}
The objectives of this lecture are:
\begin{itemize}
\item Introduce several theoretical concepts relevant to initial value problems and boundary value problems.
\item Demonstrate use of the Wronskian to determine linear independence of solutions.
\end{itemize}

\section{Initial Value Problems}

For a linear differential equation, an n\textsuperscript{th}-order initial value problem (IVP) is given by the following governing equation and initial conditions:
\begin{equation}
\text{Governing Equation: }a_n(x)\frac{d^n u}{dx^n}+a_{n-1}\frac{d^{n-1}u}{dx^{n-1}}+\cdots+a_1(x)\frac{du}{dx}+a_0(x)u=g(x)
\label{eq:ivp-ge}
\end{equation}
\begin{equation}
\text{Initial Conditions: }u(x_0)=u_0, \ u^{\prime}(x_0)=u_1,\dots,u^{(n-1)}(x_0)=u_{n-1}
\label{eq:ivp-ics}
\end{equation}
\marginnote[-2.0cm]{\textbf{Note:} for an initial value problem, all of the initial conditions are provided at the same value of $x$; in accordance to custom we call this $x_0$.  The name \emph{initial} condition gives the implication that these conditions are at some ``end'' of the interval (beginnig, left side, whatever) and in most all examples and exercises this is indeed the case.  It is \underline{not} a requirement, however.}
\noindent \newthought{We seek a} function defined on some interval containing $x_0$ that satisfies the differential equation with $n$ conditions applied.\marginnote{Generally for an n\textsuperscript{th}-order IVP you will need $n$ conditions.} The theorem below, which we will use by \emph{citing} rather than \emph{proving} gives us assurance that, subject some fairly reasonable assumptions, such a solution will exist.  

\begin{theorem}[Existence and Uniqueness for IVPs]
If $a_n(x),a_{n-1}(x),\dots,a_1(x),a_0(x)$ and $g(x)$ are continuous on an interval $\mathcal{I}$, and if $a_n(x) \ne 0$ for every $x \in \mathcal{I}$, and if $x_0$ is any point in this interval, then a solution $u(x)$ of the IVP exists on the interval and it is unique.
\label{thm:IVP-exist-and-unique}
\end{theorem}
\newthought{For this class} we will adopt a mostly operational definition of continuity: if you can draw the function throughout the specified interval without picking up your pencil or without diverging to infinity, then the function is continuous.  

Consider, as an example, the following initial value problem:
\begin{equation}
u^{\prime \prime}-4u = 12x, \ \ u(0)=4, \ u^{\prime}(0)=1
\end{equation}
This IVP satisfies the conditions of Theorem \ref{thm:IVP-exist-and-unique} since all of the coefficients and $g(x)$ are continuous and $a_1$ is constant and nonzero; hence a unique solution exists on any interval and that solution is unique.\marginnote[-1.5cm]{Take a moment to verify that $u(x)=3e^{2x}+e^{-2x}-3x$ satisfies both the governing equation and initial conditions and thus is \emph{the} unique solution to this IVP.}

Here is an IVP that does \emph{not} satisfiy the criteria of Theorem \ref{thm:IVP-exist-and-unique}:
\begin{equation}
x^2u^{\prime \prime} - 2xu^{\prime}+2u=6, \ \ u(0)=3, \ u^{\prime}(0)=1
\end{equation}
\noindent In this case, the coefficients and $g(x)$ are all continuous but $a_2(x)$ is equal to zero at $x=0$.  This might not be a problem---i.e. if $x=0$ is not in the interval of interest for the IVP then we are okay---but since $x_0=0$, $x=0$ \emph{must} be in the domain for the theorem to apply.  So we have no assurances that a solution exists or, if a solution does exist, it may not be unique.\marginnote[-1.5cm]{You should take a moment to verify that $u=cx^2+x+3$ is a solution for \emph{any} choice of parameter $c$.}

\section{Boundary Value Problems}
For this section let us, without undue loss of generality, consider a 2\textsuperscript{nd}-order boundary value problem (BVP):
\begin{equation}
\text{Govering Equation: }a_2(x)\frac{d^2u}{dx^2}+a_1(x)\frac{du}{dx}+a_0(x)u=g(x) 
\end{equation}
\begin{equation}
\text{Boundary Conditions: } y(a)=y_0, \ \ y(b)=y_1, \ \ a \ne b
\end{equation}

\newthought{Depending on the} boundary conditions, BVPs may have no solutions, one unique solution, or infinitely many solutions.

\vspace{1.0cm}
\noindent \underline{\textbf{Example:}} The equation $u^{\prime \prime}+16u = 0$ has the general solution $u(t) = c_1 \cos{(4t)}+c_2 \sin{(4t)}$.  Consider the three different sets of boundary conditions provided below.
\begin{enumerate}[label=\alph*)]
\item $x(0)=0, \ \ x(\pi/2)=0$
Application of the first boundary condition gives us $c_1(1)+c_2(0)=0 \Rightarrow c_1 = 0$.  The second boundary condition is $c_2\sin{(2 \pi)} = 0,$ which is true for \emph{any} value of $c_2$.  Therefore there problem has infinitely many solutions.
\item $x(0)=0, \ \ x(\pi/8)=0$
The first boundary condition again gives us $c_1=0$; the second condition $c_2\sin{(4 \frac{\pi}{8})}$ is only satisfied if $c_2=0$.  Thus $c_1 = c_2 = 0$; only the trivial solution, $u=0$, satisfies both the differential equation and boundary conditions.  This is not a very interesting solution but at least it \emph{is a solution} so we will take this as an example of a BVP having a unique solution.
\item $x(0)=0, \ \ x(\pi/2)=1$
In this case, again $c_1=0$ from the first boundary condition.  This leaves the second boundary condition: $c_2 \sin{\left(4 \frac{\pi}{2}\right)} = c_2(0) = 1$ which cannot be satisfied for any value of $c_2$.  In this case \emph{no} solution exists.

\end{enumerate}

\section{Superposition and Linear Dependence}
In this section some important theorems regarding IVPs and BVPs will be presented.  No attempt will be made to prove these theorems; we will simply take these theorems as facts that are relevant for this course that you should try to understand as best you can.

\begin{theorem}[Superposition Principle for Homogeneous Equations]
Let $u_1,u_2,\dots,u_k$ be solutions of a homogeneous n\textsuperscript{th}-order linear differential equation.  Then any linear combination of those solutions 
$$u = c_1u_1+c_2u_2+\cdots+c_ku_k$$where $c_1,c_2,\dots,c_k$ are arbitrary constants, is also a solution.
\end{theorem}
\marginnote[-3.0cm]{\textbf{Note:} It is essential that \emph{both} the govering equation and conditions (boundary or initial) for the linear differential equation are homogeneous.  As a reminder, this means that \emph{all} terms in the governing equation must either a) involve the dependent variable or one of its derivatives; or b) be equal to zero.} 
As an example, If I denote the linear homogeneous differential equation as $\mathcal{L}$, then $\mathcal{L}(u_i) = 0$ for any $i \in [1,2,\dots,k]$.  By the linearity property of $\mathcal{L}$, for any constants $\alpha$ and $\beta$:
\begin{align*}
\mathcal{L}(\alpha u_i + \beta u_j) &=  \alpha \mathcal{L}(u_i) + \beta \mathcal{L}(u_j)\\
&= \alpha(0) + \beta(0) \\
&= 0
\end{align*}

\begin{theorem}[Linear Dependence / Independence of Functions]
A set of functions $f_1(x),f_2(x),\dots,f_k(x)$ is said to be \emph{linearly dependent} on an interval $\mathcal{I}$ if there exist constants $c_1,c_2,\dots,c_k$, \emph{not \underline{all} of which are zero}, such that
$$c_1f_1(x) + c_2f_2(x)+\cdots+c_kf_k(x) = 0$$
\noindent for every $x \in \mathcal{I}$.  If the set of functions is \emph{not} linearly dependent, it is linearly independent.
\label{th:linear-dep}
\end{theorem} 
\marginnote[-1.75cm]{What if a member of the set of function is $f(x)=0$?}
Repeatedly throughout this course we will want to clarify whether or not, say, two functions are linearly independent of each other.  I think most engineers have a general idea of what it is we \emph{mean} when we say two functions are linearly independent or dependent but Theorem \ref{th:linear-dep} specifies what these things mean \emph{mathematically}.  
\newthought{We need a test} to help us determine if the members of a set of functions are linearly independent or not. This will be especially important as we evaluate solutions to a linear homogeneous differential equation.  Even if you are the sort of savant who can, by inspection, always detect linear dependence, you might have a hard time convincing your friends that your assessment is always correct.  Luckily, there is a theorem that provides a suitable test that can serve as irrefutable evidence of the state of linear dependence/independence of functions.

\begin{theorem}[Criterion for Linearly Independent Solutions]
Let $u_1,u_2,\dots,u_n$ be solutions of a homogeneous linear n\textsuperscript{th}-order differential equation defined on an interval $\mathcal{I}$.  Then the set of solutions is linearly indepenent on the interval if and only if the Wronskian of the solution is non-zero for every $x \in \mathcal{I}$.  
\end{theorem}
The Wronskian is a function that takes functions as arguments and returns a scalar numeric quantity.\sidenote{Sometimes such functions are referred to as \emph{functionals}.}
\begin{equation}
W(u_1,u_2,\dots,u_n)=
\begin{vmatrix}
u_1 & u_2 & \cdots & u_n \\
u^{\prime}_1 & u^{\prime}_2 & \cdots & u^{\prime}_n \\
\vdots & \vdots & \vdots & \vdots \\
u^{(n-1)}_1 & u^{(n-1)}_2 & \cdots & u^{(n-1)}_n
\end{vmatrix}
\label{eq:Wronskian-def}
\end{equation}
where $|\cdot|$ donetes the matrix determinant.  For large values of $n$ this is also difficult to calculate but, for the case $n=2$, engineering students should be familiar with the formula:
\begin{equation}
W(u_1,u_2) = 
\begin{vmatrix}
u_1 & u_2 \\
u^{\prime}_1 & u^{\prime}_2 \\
\end{vmatrix}
= u_1 u^{\prime}_2 - u^{\prime}_1 u_2
\end{equation}

\begin{definition}[Fundamental Set of Solutions]
Any set $u_1,u_2,\dots,u_n$ of $n$ linearly independent solutions of the homogeneous linear n\textsuperscript{th}-order differential equation on an interval is said to be a fundamental set of solutions on an interval $\mathcal{I}$.  
\end{definition}
\begin{theorem}[Existence of a Fundamental Set]
There exists a fundamental set of solutions for the homogeneous linear n\textsuperscript{th}-order differential equation on an interval $\mathcal{I}$.
\end{theorem}

\begin{definition}[General Solution---Homogeneous Equation]
Let $u_1,u_2,\dots,u_n$ be a fundamental set of solutions to the homogeneous linear n\textsuperscript{th}-order differential equation defined on an interval $\mathcal{I}$, then the general solution is:
$$u(x) = c_1u_1(x)+c_2u_2(x)+\cdots+c_nu_n(x)$$
\end{definition}

\newthought{It is important} to understand from the above that:
\begin{itemize}
\item \emph{any} possible solution to the homogeneous, linear, n\textsuperscript{th}-order differential equation can be constructed by setting the coefficients of the general solution; and
\item there is \textbf{no} solution that \underline{cannot} be constructed from the general solution.
\end{itemize}

