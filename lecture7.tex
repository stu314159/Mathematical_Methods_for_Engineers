\chapter{Lecture 7 - Review of Power Series}
\label{ch:lec7}
\section{Objectives}
The objectives of this lecture are:
\begin{itemize}
\item Review definitions and basic properties of power series.
\item Illustrate important basic operations on power series
\end{itemize}

\section{Introduction and Review}
The methods that we have discussed so far have largely been a review of differential equations class.  Sadly, even in the handful of lectures that we have had, our methods for solving equations are largely exhausted.  We can solve constant coefficient linear equations, and variable coefficient linear equations \emph{if} they happen to be Cauchy-Euler equations.  We can solve many first-order linear equations but if the equation is nonlinear we are sunk unless they happen to be separable.  This leaves out a lot of interesting equations.  In this sequence of lectures we will discuss how to solve linear equations with variable coefficients (other than Cauchy-Euler equations).  To do this we will need to use power series. 

\newthought{You learned about} power series back in calculus class, but you weren't ready to use them for this important application.  Now you are and now this is what we will do.  We will begin this section with some definitions that will be needed as we describe the use power series in the solution of differential equations.

\subsection{Definitions}
\begin{definition}[Sequence]
A \emph{sequence} is a list of numbers (or other mathematical objects, like functions) written in a definite order.

\begin{equation*}
\left\{c_0, c_1, c_2, c_3, \dots , c_n\right\}
\end{equation*}
\end{definition}

\index{limit}
\index{convergence}
\index{divergence}
\begin{definition}[Limit of a Sequence, convergence, divergence]
A sequence has a \emph{limit} $(L)$ if we can make the terms $c_n$ arbitrarily close to $L$ by taking $n$ sufficiently large.  If $\lim_{n\to \infty} c_n$ exists, we say the sequence \emph{converges}; otherwise, we say the sequence \emph{diverges} or is \emph{divergence}.
\end{definition}

There are various mathematical tools available for determining if an infinite sequence converges or diverges without needing to examine every element.

\index{series}
\index{infinite series}
\begin{definition}[Series, infinite series]
A \emph{series} is the sum of a sequence. For example, $S_0 = c_0$; $S_1 = c_0+c_1$; $S_n = c_0+c_1+\cdots+c_n$.  If the sequence is infinite, we call the sum an infinite series.
\end{definition}

\begin{definition}[Series Convergence]
Given a series $\sum\limits_{n=0}^{\infty}s_i=s_1+s_2+\cdots+s_n+\cdots$, let $s_n$ denote its $n$\textsuperscript{th} partial sum.  If the sequence $\left\{s_n \right\}$ is convergent then the series is convergent to the same limit.  Otherwise the series is divergent.\marginnote{We will use notation such as $s_n\to \infty$ to indicate that the partial sum is unbounded.}
\end{definition}

\index{power series}
\begin{definition}[Power Series]
A series of the form $\sum\limits_{n=0}^{\infty}c_n(x-a)^n=c_0+c_1(x-a)+\cdots$ is called a Power Series. The constant $a$ is referred to as the ``center'' of the power series.\marginnote{For almost all of the power series we will work with in this class, the series will be centered on $a=0$ and will be denoted $\sum\limits_{n=0}^{\infty} c_nx^n$.}
\end{definition}


\index{convergence, interval of}
\index{convergence, radius of}
\begin{definition}[Interval of Convergence, Radius of Convergence]
The set of all real numbers $x$ for which the series converges.  This interval can also be expressed as a \emph{radius of convergence.} ($R$); the series converges for all $a-R < x < a+R$.
\end{definition}

\subsection{Ratio Test}
We should have at least one test that we can use to decide whether or not a series, at least a power series, converges.  The test we will use is called the \emph{Ratio Test}; so named because it involves the ratio of the $n$\textsuperscript{th} and $(n+1)$\textsuperscript{th} term in a power series.  The Ratio Test is shown in Equation \ref{eq:ratio-test}.

\begin{equation}
\lim_{n \to \infty}\left|\frac{c_{n+1}(x-a)^{(n+1)}}{c_n(x-a)^n} \right| = \left|x-a \right| \lim_{n \to \infty} \left|\frac{c_{n+1}}{c_n} \right| = L
\label{eq:ratio-test}
\end{equation}
The following cases are considered:
\begin{itemize}
\item if $L<1$ then the series converges absolutely.\marginnote{\textbf{Note:} \underline{absolute convergence} means that the series converges irrespective of the signs of each term. (i.e. whether or not all terms are positive, negative, or a mix of both positive and negative.)}
\item if $L = 1$ then the test is inconclusive; some other test must be used; and
\item if $L > 1$ then the series diverges.
\end{itemize}

\vspace{0.5cm}

\noindent\textbf{Example: } Find the radius of convergence and associated interval of convergence for the following power series:

\begin{enumerate}
\item $\sum\limits_{n=0}^{\infty} (-1)^nx^n$
\begin{align*}
\lim_{n \to\infty} \left|\frac{c_{n+1}(x-a)^{n+1}}{c_n(x-a)^n} \right| = |x-a| \lim_{n \to\infty} \left|\frac{c_{n+1}}{c_n} \right| = L &< 1, \ \ a=0, \ c_n=(-1)^n \\
\lim_{n\to\infty}\left|\frac{x^{n+1}}{x^n}\right| &< 1 \\
\left| x \right| \lim_{n\to\infty} \left| 1 \right| &< 1 \\
\Rightarrow |x| &< 1
\end{align*}
The radius of convergence $R=1$ and the interval of convergence is $x\in(-1,1)$.\marginnote[-1.0cm]{Here I have purposely avoided analyzing the end-points to see if we could use a closed or partially-closed interval instead.  Since we limited $L<1$, that only gives the radius of absolute convergence.  If we wanted to be picky, we could allow $L=1$ and use some other test to determine if the series converges.  If we did that in this case we would find that the series diverges at both endpoints.}  

\vspace{4.0cm}

\item $\sum\limits_{n=0}^{\infty} \frac{(-1)^n x^n}{n(n+1)}$
\begin{align*}
\lim_{n \to\infty} \left|\frac{c_{n+1}(x-a)^{n+1}}{c_n(x-a)^n} \right| = |x-a| \lim_{n \to\infty} \left|\frac{c_{n+1}}{c_n} \right| = L &< 1, \ \ a=0, \ c_n=\frac{(-1)^n}{n(n+1)} \\
\lim_{n \to \infty} \left|\frac{x^{n+1}}{(n+1)(n+1+1)}\frac{n(n+1)}{x^n}  \right|&<1 \\
|x| \lim_{n \to\infty}\left|\frac{n}{n+2} \right| &<1 \\
|x| \cancelto{1}{\lim_{n \to\infty}\left|\frac{n}{n+2} \right|}&<1 \\
|x| &< 1
\end{align*}
Once again, the radius of convergence $R=1$ and the interval of convergence is $x\in(-1,1)$.

\vspace{4.0cm}

\item $\sum\limits_{n=1}^{\infty} \frac{x^{2n}}{2^nn^2}$

\begin{align*}
\lim_{n \to\infty} \left|\frac{c_{n+1}(x-a)^{n+1}}{c_n(x-a)^n} \right| = |x-a| \lim_{n \to\infty} \left|\frac{c_{n+1}}{c_n} \right| = L &< 1, \ \ a=0, \ c_n=\frac{1}{2^n n^2} \\
\lim_{n \to\infty}\left|\frac{x^{2n+2}}{2^{n+1}(n+1)^2} \frac{2^n n^2}{x^{2n}} \right| &< 1 \\
\lim_{n \to\infty} \left|\frac{x^2}{2} \frac{n^2}{(n+1)^2} \right| &< 1 \\
\frac{\left|x^2 \right|}{2} \cancelto{1}{\lim_{n \to\infty}\left|\frac{n^2}{(n+1)^2} \right|} &< 1 \\
\left|x^2\right| &< 2 \\
|x| &< \sqrt{2}
\end{align*}
In this case the radius of convergence $R=\sqrt{2}$ and the interval of convergence is $x\in(-\sqrt{2},\sqrt{2})$.\marginnote{In this case, more detailed analysis shows that this series converges at both endpoints; a closed interval could be used instead.}

\vspace{2.0cm}

\item $\sum\limits_{n=1}^{\infty} \frac{\left(x-2 \right)^n}{3^n}$ 
\begin{align*}
\lim_{n \to\infty} \left|\frac{c_{n+1}(x-a)^{n+1}}{c_n(x-a)^n} \right| = |x-a| \lim_{n \to\infty} \left|\frac{c_{n+1}}{c_n} \right| = L &< 1, \ \ a=2, \ c_n=\frac{1}{3^n} \\
\lim_{n \to\infty} \left|\frac{\left(x-2\right)^{n+1}}{3^{n+1}} \frac{3^n}{\left(x-2 \right)^n}\right| = L &< 1 \\
\frac{\left|x-2 \right|}{3} \lim_{n \to\infty}\left|1 \right| &< 1 \\
|x-2| &< 3
\end{align*}
So for this example the radius of convergence is $R=3$ about the center at $x=2$; and the interval of convergence is $x \in (-1,5)$\marginnote{For the interested reader, it can be shown that this series is divergent at both endpoints so it should remain an open interval.}
\end{enumerate}

\section{Properties of Convergent Series}
Within the radius of convergence, a power series defines a function.  Within the interval of convergence the function so defined is:\marginnote{If $x$ is not within the interval of convergence for a series or if the series is divergent then \emph{\textbf{none}} of these are true.  This is why it is important to be able to find the interval/radius of convergence.}
\begin{itemize}
\item continuous 
\item differentiable (term-by-term); and
\item integrable (term-by-term)
\end{itemize}


\begin{definition}[Identity Property for a Power Series]
If $\sum\limits_{n=0}^{\infty}c_n(x-a)^n=0, \ \ \text{R}>0$, for all numbers $x$ in the interval of convergence then $c_n=0$ for all $n$.
\end{definition}
\marginnote[-1.75cm]{Hopefully this definition seems obvious to you.  You will find that most of what we do when using power series to solve homogeneous linear differential equations is carry out the necessary algebra to ensure that the coefficients for some series all are equal to zero. }


\begin{definition}[Analytic Function]
a function $f$ is \emph{analytic} at a point $a$ if it can be represented by a power series in $x-a$ with a positive radius of convergence.
\end{definition}
\marginnote[-0.75cm]{This is just a vocabulary term that you should know.}

\subsection{Some Common Power Series}
You have probably had some exposure to power series in your previous mathematical courses.  As a reminder, I've included the power series representations of some important/common functions.

\begin{enumerate}

\item $\sin{x} = x - \frac{x^3}{3!} + \frac{x^5}{5!} - \cdots$

\item $\cos{x} = 1 - \frac{x^2}{2!} + \frac{x^4}{4!} - \frac{x^6}{6!} + \cdots$

\item $\ln{x} = \frac{x-1}{x} + \frac{(x-1)^2}{2x^2} + \frac{(x-1)^3}{3x^3} + \cdots$ 

\item $e^x = 1 + \frac{x}{1!} + \frac{x^2}{2!} + \cdots$
\end{enumerate}


\section{Combining Power Series}
This is a practical ``utility skill'' that you will need to master in order to be successful at this portion of the course.  What we need to be able to do is combine multiple power series into a single expression.  

\newthought{For example}, consider the two power series below that we want to write as a single power series:

\begin{equation*}
\sum\limits_{n=2}^{\infty} n(n-1)c_n x^{n-2} - \sum\limits_{n=0}^{\infty} c_n x^{n+1}
\end{equation*}

If I want to combine these series, I need to overcome two issues:
\begin{enumerate}

\item the powers of $x$ in each term in both summations need to be ``in phase'' -- that is the corresponding terms need to have the same power of $x$.  The first term in the first summation is constant ($x^0$) while the first term in the second summation is linear ($x^1$); and \marginnote[-1.5cm]{It is not only the first term that is important but if you can get the summations in phase for the first term, and if the power of $x$ increases by one with each consecutive term, then if the first term is correct, they will all be correct.}


\item the first summation index starts at $n=2$ while the second summation index starts at $n=0$.

\end{enumerate}


\newthought{We will address} these issues one at a time, starting with the first one.  We will leave the summation whose first term is highest order as-is; for all other summations (i.e. if there are more than two) we will ``peel-off'' any lower-order summation terms.

In this case that means we will ``peel-off'; the constant term from the first summation:

$$\underbrace{(2)(1)c_2x^0}_{\text{constant term}} + \underbrace{\sum\limits_{n=3}^{\infty}}_{\text{now }n=3} n(n-1)c_nx^{n-2} + \sum\limits_{n=0}^{\infty}c_nx^{n+1}$$
Notice that now the summation index for the first summation starts at $n=3$; this is because we've separated out the first term corresponding to $n=2$.  The two remaining summations are ``in phase'' since all of the terms now have the same power of $x$.

\newthought{The second problem} will be fixed by establishing a new common index, $k$, and re-write the existing indices ($n$ for both summations) in terms of $k$.  In each case we will set $k$ equal to the exponent of $x$ appearing in the summation.
\begin{itemize}
\item For the first summation---$\sum\limits_{n=3}^{\infty} n(n-1)c_nx^{n-2}$---we set $k=n-2$ because that is the exponent for $x$. We need to eliminate each occurrence of $n$ in the summation and replace it with it's equivalent expression in terms of $k$.  From our definition of $k$ for this summation, $n=k+2$. Our summation now can be written:\marginnote{Everywhere you see an $n$ in the original summation, replace it with a $k+2$ and simplify.}
$$\sum\limits_{k=1}^{\infty} (k+2)(k+2-1)c_{k+2}x^k$$ 

\item For the second summation---$\sum\limits_{n=0}^{\infty} c_n x^{n+1}$---we set $k=n+1$ because that is the exponent for $x$ in this summation.  This means $n=k-1$; substituting that expression in our summation gives us:
$$\sum\limits_{k=1}^{\infty}c_{k-1}x^{k}$$
\end{itemize}

With these changes our original summation can be written:\marginnote{Notice that I've made some obvious simplifications in the constant term and first summation.}
\begin{equation}
2c_2 + \sum\limits_{k=1}^{\infty} (k+2)(k+1)c_{k+2}x^k + \sum\limits_{k=1}^{\infty}c_{k-1}x^{k}
\end{equation}

The two summations are now ready to be joined into one as shown in Equation \ref{eq:summation-joined}
\begin{equation}
2c_2 + \sum\limits_{k=1}^{\infty} \left[(k+2)(k+1)c_{k+2} + c_{k-1}\right] x^{k}
\label{eq:summation-joined}
\end{equation}

